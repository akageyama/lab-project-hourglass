\documentclass[dvipdfmx]{article}
\usepackage{siunitx}
\usepackage[dvipdfmx]{graphicx}
\usepackage{bm}
\usepackage{indentfirst}
\usepackage{amsmath,amssymb}
\usepackage{color}

\renewcommand{\figurename}{図}
\renewcommand{\tablename}{表}
\renewcommand{\refname}{参考文献}
\renewcommand{\abstractname}{要旨}
\usepackage{snotez}
\setlength{\marginparwidth}{2cm} 
\setlength{\marginparsep}{1.0cm} 


\title{砂時計の重さ\\[1em]
        Weight of Hourglass}

\author{	陰山 聡\thanks{神戸大学システム情報学研究科},
		中島 涼輔\thanks{神戸大学システム情報学研究科},
		中戸 昂明\thanks{神戸大学システム情報学部 }\\[1em]
		Akira Kageyama\thanks{Graduate School of System Informatics, Kobe University, Japan}, 
		Ryosuke Nakashima\thanks{Graduate School of System Informatics, Kobe University, Japan}, and
		Komei Nakato\thanks{Department of System Informatics, Kobe University, Japan}		
		}
\date{\today}

\begin{document}

\color{magenta}
原稿 ver.~250923: シミュレーション学会和文論文誌に投稿予定
\vspace{1em}
\color{black}

%=========================================================================
\begin{abstract}
砂時計の砂が落ちている間、砂時計の重さはわずかに重くなることが知られている。
その重さの変化はとても小さいので、この効果をシミュレーションで定量的に再現するには工夫が必要である。
本研究では、砂時計内部の砂を質量が変動する物体からなる系とみなし、
運動方程式に基づいて砂時計が重くなる要因を明らかにした上で、
砂粒の鉛直方向の運動に注目した1次元シミュレーションモデルを構築した。
このシミュレーションは、ソフトスフィアモデルに基づいた個別要素法シミュレーションである。
Processing言語で実装したこのシミュレーションはノートPCで実行することが可能である。


\vspace{1em}
It is known that the weight of an hourglass increases slightly while the sand is falling. This weight change is so small that quantifying the effect in simulations is difficult. We first identify the essential factors that lead to the weight increase by modeling the sand in the hourglass as a system of varying-mass objects. We then developed a discrete element method simulation model based on the soft sphere model. Having implemented the model in the Processing language, we run this simulation on a notebook PC.

\end{abstract}
%=========================================================================

\vspace{2em}

\paragraph{キーワード}
砂時計、力学、個別要素法、可変質量系\\[1em]
Hourglass, Mechanics, Discrete Element Method, Variable mass system


\maketitle



%=============================================
\section{はじめに}
%=============================================
砂時計を手に持ち、目をつむって重さを感じてみよう。
砂時計の砂が落ち始める前の砂時計全体の重さと、
砂がすべて落ちきった後の砂時計全体の重さはもちろん変わらない。
では、砂が落ちている最中の砂時計の重さはどうであろうか。
この問題は古くから知られていて、
砂時計はわずかに重くなるが、それは人の出では感じられないほどわずかな差であることがわかっている。


Scientific American誌の有名なMathematical Gamesの連載記事でGardnerが砂時計を使ったおもちゃを紹介したときがこの問題に注目を集めた最初であろう\cite{Gardner1966-eu}。
その後すぐに、落下した砂が砂時計の底面に与える力積(つまり砂時計を重くする効果)が、
落下中の砂の分だけ砂時計の重さが軽くなる効果とちょうどキャンセルするという指摘がされた\cite{Reid1967-jq}。
そこでは、この問題は力学教育のよい題材になるであろうとも指摘された。
その議論は以下のように簡単なものである:
砂時計のオリフィスから質量$m$の砂が落下し、$\tau_f$だけの時間をかけて下面に到達するあいだ、重力$mg$の分だけ砂時計は軽くなる。
1秒間に$\mu$個の砂が落ちる砂時計であれば$\mu \tau_f$個の砂が浮いているので、砂時計は$\mu \tau_f mg$だけ軽くなる。
一方、一つの砂が底面到達したときにもつ運動量は$mg\tau$なので、落下する砂が1秒あたりに底面に与える力積、つまり力は$m\mu g\tau_f$であり、浮いている砂の重力とちょうどキャンセルする\cite{Kassandrov2023-tn}。
この議論は直感的で、説得力があるので、その後、多くの人が、砂時計の重さは変わらないと信じてきた。


だが、砂の重心に注目すると、
砂が落ちている間、砂時計の重さが変化しないはずはないと思えてくる。
最初の砂が落ち始める前、砂時計の上部で静止状態にあった重心は、
砂が落ち始めると下部に移動しはじめ、全ての砂が落ちきった後には砂時計の下部で静止状態になる。
砂時計の砂を入れたガラス部分が(ひいては砂時計を乗せた手が)下に「落ちて」きた重心を受け止めて、静止させたのだから、少なくともその間は上向きの力、つまり重みが増えたはずである。
結局、砂時計の重さの変化は、砂時計が砂を支える抗力の変化であり、重心移動の加速度に比例するはずである。
ただし、重心移動の距離はわずか(最大でも砂時計の高さ程度)であり、
しかもかなり時間をかけたゆっくりとした移動(砂時計が3分計であれば3分間)なので、
その加速度は必然的に小さく、したがって砂時計全体の重さに与える影響は小さいことは予想できる。
Shen等はガラスの粉末を使った「砂」時計の重量を測定してガラス粉末の総質量の$10^{-3}$倍程度というわずかな割合だけ実際に重くなることを示した\cite{ShenUnknown-pk}。
% - % - % - % - % - % - % - % - % - % - % - % - % - % - % - % - % - % - % - % - % - % - % - % - 
%  メモ:彼らの論文には総質量は明記されていない。
%        時間が10s から15sで、砂のfluxが3.7 g/s であるという記述があるので、
%        総質量は 35g から55.5 g と推測される。
%        重さの変化は、理論予想が49~\si{mg}に対して
%        実験結果は$30\sim 80$~\si{mg}の範囲と書いてあるので10^{-3}とした。
% - % - % - % - % - % - % - % - % - % - % - % - % - % - % - % - % - % - % - % - % - % - % - % - 


砂の代わりに金属の粉末を使った実験で、Sack等は砂時計の重さの時間変化を詳細に解析した\cite{Sack2017-rq}。
時間変化は主に3つの時期に分けることができる。
最初期、砂が落ち始めた直後から最初の砂粒が下面に到達するまでの短い間、砂時計の重量は軽くなる。
これは空中にいる砂粒の重量分だけ重みがかからないためである。
逆に最終期、最後の砂粒が落ちている短い間、砂時計は重くなる。
これは下面を叩く砂粒達が与える力積をキャンセルするだけの数の砂が空中にないためである。
この二つ短い遷移時間の間、砂時計の重量がわずかに重くなる時期が続く。


砂時計の重さの問題は、単純そうでありながら、実際にはそうでなく、
また、質量が時間変化する系の典型例として教育的にも重要であると指摘されている\cite{Kassandrov2023-tn}。


砂時計のガラスがくびれているのは、単位時間あたりに落下する砂の数の一定で小さい値にするためである。
落下する砂粒の列が一本だけであるため、落下前の砂粒も、落下後の砂粒も複雑な3次元運動をするが、
砂時計の重さの変化に寄与するのは砂の鉛直方向の運動だけである。
この複雑さを回避するするために、Tuinstra等は単一のオリフィスではなく多数(230個)の穴を持つ「ふるい」で区切られた円筒状の砂時計を作成し、
(本物の)砂$1.6$~\si{kg}を使った実験を行い、理論から予想される重さの増加を得た\cite{Tuinstra2010-wk}。
% - % - % - % - % - % - % - % - % - % - % - % - % - % - % - % - % - % - % - % - % - % - % - % - % - % - % - % - % - % - % - % - 
% 理論的には$0.116$~\si{g}の重さの増加が期待され、測定結果は$0.121$~\si{g}であった。
% - % - % - % - % - % - % - % - % - % - % - % - % - % - % - % - % - % - % - % - % - % - % - % - % - % - % - % - % - % - % - % - 


本研究においても、砂粒が鉛直方向にのみ運動するという理想化を行う。
これにより理論解析も容易になる。


コンピュータシミュレーションでこの効果、つまり砂が落下している最中の砂時計がわずかに重くなる効果を実証した研究はまだないようである。
本研究の目的は、落下中の砂の分だけ軽くなる効果と落下した砂が与える力積がちょうどキャンセルするという基本的な効果を再現することはもちろん、
砂の重心移動によるわずかな重さの変化という微弱な効果も定量的に再現するコンピュータシミュレーションを行うことである。
このシミュレーションはプログラミング初心者にも扱えるようデザインされたProcessing言語\footnote{\texttt{https://processing.org}}で書かれており、
ノートPCで実行可能なほど小規模なシミュレーションなので、
この興味深い力学問題を直接体験することができる。


%=============================================
\section{理論}
%=============================================

%------------------------------------------
\subsection{問題設定} \label{250919151350}
%------------------------------------------

$T_0$秒間を測る砂時計を考える。
重力加速度を$g$、砂粒の総数を$N_0$、砂全体の質量を$M_0$とすると、
砂粒一つの質量$m$は、
%==========
\begin{equation} \label{250517100232} 
   m = \frac{M_0}{N_0} 
\end{equation}
%==========
である。
1秒間に落ちる砂粒の数を
%==========
\begin{equation} \label{250515082817} 
   \mu = \frac{N_0}{T_0}
\end{equation}
%==========
とする。
$m$と$\mu$は


鉛直上向きに$y$軸をとり、
砂が落ちる位置すなわちオリフィスの位置を$y=0$とする。
最初の砂粒が落下する距離、つまり砂時計の底面の位置を$y=-H_0$とする。


% . . . . . . . . . . . . . . . . . . . . . . . . . . . . . . . . . . . . . . . . 
      \marginnote{図\ref{250919103231}}
% . . . . . . . . . . . . . . . . . . . . . . . . . . . . . . . . . . . . . . . . 
砂時計の床面に働く抗力を求めることを目標として、運動方程式を立てる。
図\ref{250919103231}に示すように、
オリフィスよりも上にある砂の層を質量$M_1(t)$をもつ物体1、
下に落下して静止した砂の層を質量$M_2(t)$をもつ物体2、
落下中の砂粒全体を質量$M_3(t)$をもつ物体3とみなす。
つまり可変質量物体が相互作用する力学の問題として砂時計の重さを扱う。



簡単のため、オリフィス付近を除いた砂時計の断面積は一定とする。
計時を開始する前の砂の層の厚さを$K_0$、
砂粒の高さあたりの質量密度を$\sigma$とすると
%==========
\begin{equation} \label{250703152145} 
   \sigma = \frac{M_0}{K_0} %= \frac{mN_0}{K_0} 
\end{equation}
%==========
である。
%==========
\begin{equation} \label{250919090329} 
   \frac{m\mu}{\sigma} = \frac{K_0}{T_0}
\end{equation}
%==========
という関係がある。

%------------------------------------------
\subsection{物体1(落下前の上の砂層)}
%------------------------------------------

計時開始時刻を$t=0$とすると、
時刻$t$では$\mu t$個の砂粒がオリフィスから落下しているので、
この時刻にオリフィスの上の部分にある砂、つまりまだ落下していない砂の数は
%==========
\begin{equation} \label{250703152453} 
   N_1(t) =  N_0 - \mu t
\end{equation}
%==========
である。
したがってオリフィス上部の砂の質量は
%==========
\begin{equation} \label{250703152655} 
   M_1(t) = m (N_0-\mu t)
\end{equation}
%==========
である。
この時刻におけるオリフィス上部の砂層の厚さは
%==========
\begin{equation} \label{250703152748} 
   K_1(t) = \frac{M_1(t)}{\sigma} = \frac{m}{\sigma} (N_0-\mu t)
\end{equation}
%==========
である。
上の層の厚さは一定の割合で減少する。
%==========
\begin{equation} \label{250919084914} 
   \dot{K}_1(t) = -\frac{m\mu}{\sigma}
\end{equation}
%==========
この層の重心の$y$座標を$Y_1(t)$とすると
%==========
\begin{equation} \label{250703153013} 
   Y_1(t) = \frac{K_1(t)}{2} = \frac{m}{2\sigma} (N_0-\mu t)
\end{equation}
%==========
物体1の運動量は
%==========
\begin{equation} \label{250917102959} 
   P_1(t)  = M_1(t) \dot{Y}_1(t) 
\end{equation}
%==========
その時間微分は
%==========
\begin{equation} \label{250919104731} 
   \dot{P}_1(t) = \dot{M}_1(t)\dot{Y}_1(t) = \frac{m^2\mu^2}{2\sigma}
\end{equation}
%==========
である。
運動量$P_1$は下向き($P_1<0$)であるが、その時間変化$\dot{P}_1$は上向き($\dot{P}_1>0$)である。


%------------------------------------------
\subsection{物体2(落下後の下の砂層)}
%------------------------------------------

同様に下の砂層の厚さを$K_2(t)$、質量を$M_2(t)$とする。
これらの量が正になるのは時刻$t=0$に落下した最初の砂粒が下の床に到達するまでの時間、
つまり自由落下時間
%==========
\begin{equation} \label{250703153227} 
   \tau_{f0} = \sqrt{\frac{2H_0}{g}}
\end{equation}
%==========
以降である。
時刻$t>\tau_{f0}$では下の層に砂粒が$\mu$の割合で降り注ぐので、
%==========
\begin{equation} \label{250919091147} 
   M_2(t) = m\mu (t-\tau_{f0})
\end{equation}
%==========
と書ける。
実がこれは正確ではないが、その誤差は小さい。詳細はAppendix~\ref{250919092352}参照。


下の砂層の厚さは
%==========
\begin{equation} \label{250919104425} 
   K_2(t) = \frac{M_2(t)}{\sigma} = \frac{m\mu}{\sigma}(t-\tau_{f0})
\end{equation}
%==========
その時間微分は
%==========
\begin{equation} \label{250920171606} 
  \dot{K}_2(t)   = \frac{m\mu}{\sigma}
\end{equation}
%==========
である。
下の砂層の重心の$y$座標は
%==========
\begin{equation} \label{250703154424} 
   Y_2(t) = -H_0 + \frac{K_2(t)}{2} 
\end{equation}
%==========
である。
物体2の運動量は
%==========
\begin{equation} \label{250917103019} 
   P_2(t) = M_2(t)\dot{Y}_2(t)  
\end{equation}
%==========
その時間微分は
%==========
\begin{equation} \label{250919104243} 
   \dot{P}_2(t) = \dot{M}_2(t)\dot{Y}_2(t) = \frac{m^2\mu^2}{2\sigma}
\end{equation}
%==========
である。


Appendix~\ref{250919092352}にしめすように、式\eqref{250920171606}には誤差がある。
正しくは
%==========
\begin{equation} \label{250920171723}
   \dot{K}_2  = \frac{m\mu}{\sigma}\left( 1+ \frac{\epsilon}{2\sqrt{1-\beta  (t-\tau_{f0})}}\right)+ \mathcal{O}(\epsilon^2)
\end{equation}
%==========
である。ここで
%==========
\begin{equation} \label{250919092251} 
   \epsilon = \frac{K_0/T_0}{\sqrt{2gH_0}}  \ll 1
\end{equation}
%==========
は微小量である。(Appendix~\ref{250919092352}参照)。
運動量の時間変化に与えるこの誤差の影響は小さい。
%==========
\begin{equation} \label{250920161509} 
    \dot{P}_2 = \dot{M}_2\frac{\dot{K_2}}{2} = \frac{\sigma}{2}(\dot{K}_2)^2 
    =  \frac{\sigma}{2} \frac{m^2\mu^2}{\sigma^2}\left(1+ \frac{\epsilon}{\sqrt{1-\beta t}}\right) + \mathcal{O}(\epsilon^2)
\end{equation}
%==========
%==========
\begin{equation} \label{250923165505}
    \dot{P}_2
    	 =  \frac{m^2\mu^2}{2\sigma}\left(1 + \frac{\epsilon}{\sqrt{1-\beta t}}\right) + \mathcal{O}(\epsilon^2)
\end{equation}
%==========


%------------------------------------------
\subsection{物体3(落下中の砂粒)}
%------------------------------------------

$t\ge \tau_{f0}$のある時刻$t$に落下している最中の砂粒は、
高さ
%==========
\begin{equation} \label{250919083726} 
   K_3(t) = H_0 - K_2(t)
\end{equation}
%==========
に分布している。


物体3の運動量を評価するためにまず、物体3内部の運動量分布は高さ$y$によらず一定であることに注意する。
位置$y$を速度$v_y(y)$で下方$(v_y<0)$に通過する砂粒の数は常に$\mu$個なので、砂粒の数密度を$n(y)$とすると、
%==========
\begin{equation} \label{250917101402} 
   n(y) v_y(t) = -\mu = \text{const.}
\end{equation}
%==========
である。したがって物体3の運動量の密度は
%==========
\begin{equation} \label{250917102631} 
   p_3 = mn(y)v_y(y)= -m \mu = \text{const.}
\end{equation}
%==========
は一定である。
したがって物体3の運動量は
%==========
\begin{equation} \label{250917102719} 
   P_3(t) =p_3 K_3(t) =  -m\mu(H_0-K_2(t))
\end{equation}
%==========
である。
その時間変化は
%==========
\begin{equation} \label{250919105348} 
   \dot{P}_3(t)= m\mu\dot{K}_2(t) = \frac{m^2\mu^2}{\sigma}
\end{equation}
%==========
である。


$\mathcal{O}(\epsilon)$の補正の影響は小さい。
%==========
\begin{equation} \label{250920161728} 
   \dot{P}_3 = m\mu \dot{K}_2 = \frac{m^2\mu^2}{\sigma}\left( 1 + \frac{\epsilon}{2\sqrt{1-\beta t}}\right) + \mathcal{O}(\epsilon^2)
\end{equation}
%==========
%したがって
%%==========
%\begin{equation} \label{250920162242} 
%   \dot{P}_2 + \dot{P}_3 =  \frac{m^2\mu^2}{\sigma} \left( \frac{3}{2} + \frac{\epsilon}{\sqrt{1-\beta t}}\right) + \mathcal{O}(\epsilon^2)
%\end{equation}
%%==========


%------------------------------------------
\subsection{運動方程式}
%------------------------------------------


$F_1 (>0)$を砂時計の構造体が(オリフィスのある面で)物体1を支える抗力、
$F_2 (>0)$を砂時計の構造体が(底面で)物体2を支える抗力とする。
この二つの和$F=F_1+F_2$は砂時計の構造体が砂を支える力である。
砂が落ちている間に$F$が砂にかかる重力$M_0 g$よりも大きければ砂時計は重くなる。


物体1と物体3の間に作用する力を$F_{13}$、物体2と物体3の間に作用する力を$F_{23}$とすると、
この系の運動方程式は
%==========
\begin{align}
  \dot{P}_1(t) &=   - M_1(t) g + F_{13} + F_1\label{250703155015a}  \\
  \dot{P}_2(t)  &=  - M_2(t) g  - F_{23} + F_2 \label{250703155015b} \\
  \dot{P}_3(t) &=  - M_3(t) g  - F_{13} + F_{23} \label{250703155015c} 
\end{align}
%==========


上の3つの式の両辺を足し、
式\eqref{250919104731}, \eqref{250919104243}, \eqref{250919105348}を使うと
%==========
\begin{equation} \label{250917103450} 
  2\frac{m^2\mu^2}{\sigma}=  -(M_1+M_2+M_3) g + F_1 + F_2
\end{equation}
%==========
したがって
%==========
\begin{equation} \label{250919105951} 
   F = F_1 + F_2 = M_0 g +  2\frac{m^2\mu^2}{\sigma}
\end{equation}
%==========
つまり計時中の砂時計が砂を支える抗力は、右辺第2項
%==========
\begin{equation} \label{250703173105} 
   \Delta F  = 2 \frac{m^2\mu^2}{\sigma}  = 2 \frac{M_0K_0}{T_0^2} 
\end{equation}
%==========
だけ大きくなる。
最後の式変形では式\eqref{250919090329}と$m\mu=M_0/T_0$を使った。
力ではなく重さで表現すれば
%==========
\begin{equation} \label{250703173052} 
    \Delta W  =  2 \frac{M_0K_0}{gT_0^2} 
\end{equation}
%==========
である。


落下する砂粒が床に与える力積が、空中にある砂粒の重量とキャンセルするので砂時計の重さは変わらない、という冒頭に述べた議論は、
運動方程式\eqref{250703155015a}--\eqref{250703155015c}において、
左辺と$F_{13}$を無視した上で、$F_{23}=m\mu g\tau_f$としたことに相当する。


砂時計の重さの変化は、$H_0$(砂の落下距離)には依存しない。
計時中の砂時計の重心は下に移動し、
その速度は下向きであるが、
加速度が上向きであることがこの重さの起源である。
つまり重心の軌跡を時間の関数としてプロットすると、自由落下の場合のような上に凸の放物線ではなく、下に凸な放物線になる。

%
%Tuinstra等はくびれのない円筒形の容器を作り、その真ん中に設置した多数の穴をもつ「ふるい」状の板を通過して落下する砂柱の同時落下で実験を行った\cite{Tuinstra2010-wk}。
%その実験では$M_0=1.6~\si{(kg)}$、$T_0=23.6~\si{(s)}$、$K_0=0.198~\si{(m)}$である。
%%
%% --------------------------------------------------------------------------------------------------------------------------------
%% メモ:彼らの論文では 
%%      M = 1.6 (kg)
%% の砂を使ったとかかれているがTとK (彼らはLと書いている)の値は明記されていない。その代わり、測定値から
%%     M/T = 67.8 \times 10^{-3} (kg/s)
%%     L/T = 0.84 \times 10^{-2} (m)
%%  と書かれている。L/Tは時間の経過にともなって砂の層が薄くなるその速さを測定している。
%%  同様にM/Tも実測している。巧妙である。この3つの値からM, L(=K_0), Tを計算したのが上の値である。
%% --------------------------------------------------------------------------------------------------------------------------------
%式\eqref{250703173052}から評価される重さの変化は$\Delta W = 1.16\times 10^{-4}~\si{(kg)}$である。
%彼らの測定値は$\Delta W = (1.21\pm 0.06)\times 10^{-4}~\si{(kg)}$であった。
%相対的にこれほど微小な重さの差を検出するためには十分高い精度をもつ重量計と洗練された実験技術が必要である。



%=============================================
\section{シミュレーション}
%=============================================
%------------------------------------------
\subsection{砂粒のモデル}
%------------------------------------------

我々は粒子法(個別要素法)に基づくシミュレーションによって砂時計の重さの変化を再現する。
第\ref{250919151350}節で述べた理論は、砂粒の水平方向の運動は全く寄与しておらず、この現象は本質的に1次元的運動であることから、このシミュレーションにおいても1次元モデルを採用する。
落下する砂は複数の本数があるが、それぞれの砂柱は常に鉛直方向に1次元的な運動をする。
下の砂層に到達した砂粒は水平方向に動くことはなく、落ちた順番に上に積み上がっていく。
砂粒同士の接触相互作用は上下二つの隣の砂粒とだけ考えればよい。


砂粒同士の相互作用および砂粒と床面との相互作用はソフトスフェアモデルで表現する。
一つの砂粒を半径$a$、質量$m$の球とする。
$i$番目の砂粒と、下方にいる隣の砂粒$i-1$との距離が$2a$よりも大きい場合は互いに力をおよぼすことはない。
各砂粒は自然長$a$、ばね定数$k$のバネがあるものとする。
したがって粒子間距離が$2a$未満になった場合にバネによる反発力が生じる。
また、粒子同士の相対速度の大きさ$v$に比例して$bv$の抵抗力が生じるダンパーを想定する。


砂時計の中の砂粒が砂時計の床面、あるいは他の砂粒に高速で衝突したときもほとんど跳ね返ることなく静止する。
完全非弾性衝突に近いこのような現象をソフトスフェアモデルで再現するためには適切な$k$と$b$を設定するだけでなく、
十分に細かい時間刻み幅$\Delta t$をとる必要がある。
本シミュレーションでは時間積分法として陽的4次ルンゲ・クッタ法を採用した。


砂時計のオリフィスの位置から床面までの距離を$H_0$とする。
最初の砂粒は高さ$H_0$から自由落下し、速度
%==========
\begin{equation} \label{250919160111} 
   v_{f0} = \sqrt{2gH_0}
\end{equation}
%==========
で砂時計の床面に衝突し、静止する。
ここで$g$は重力加速度である。
本シミュレーションでは$g=9.80665$~\si{(m/s)}とした。
砂粒の半径と自由落下速度できまる時間スケールは
%==========
\begin{equation} \label{250919160801} 
   \tau_{f0} = \frac{a}{v_{f0}}
\end{equation}
%==========
である。
時間刻み幅は
%==========
\begin{equation} \label{250919160045} 
   \Delta t = \alpha \tau_{f0}
\end{equation}
%==========
として決める。
ここで係数$\alpha$は十分小さい値で、本シミュレーションでは$\alpha=10^{-3}$とした。
バネ定数$k$は単振動の周期が$\tau_{f0}$と等しくなるよう
%==========
\begin{equation} \label{250919161147} 
   k = m\left(\frac{2\pi}{\tau_{f0}}\right)^2
\end{equation}
%==========
と設定する。
ダンパーの減衰係数は定数$\beta$を
%==========
\begin{equation} \label{250919162003} 
   b = \beta b_0
\end{equation}
%==========
として与える。
ここで$b_0=2\sqrt{mk}$は臨界減衰係数で、
$\beta$は定数である。
本シミュレーションでは$\beta=10$とした。


% . . . . . . . . . . . . . . . . . . . . . . . . . . . . . . . . . . . . . . . . 
      \marginnote{図\ref{250921122102}}
% . . . . . . . . . . . . . . . . . . . . . . . . . . . . . . . . . . . . . . . . 
図\ref{250921122102}に半径$a=1\times 10^{-2}$~\si{(m)}の粒子を高さ1~\si{(m)}から自由落下させ、床面と衝突して静止するテストの結果を示す。
この図の左のグラフが示すように、粒子が放物線を描いて落下し、$y=0$に想定した床面に衝突して跳ね返ることなく止まる。
右のグラフは衝突の瞬間近くの拡大図である。
粒子の中心位置が$y=a$(右のグラフの灰色水平線)に達したときに床面と接触する。
この計算の時間刻み幅は$\Delta t = 2.2580\times 10^{-6}$~\si{(s)}である。



%------------------------------------------
\subsection{砂時計のモデル}
%------------------------------------------
砂時計は二つの床面をもつ「2階建て」構造をもつ。
床面は$y=0$と$y=-H_0$にあり、ここでは前者を$U_F$ (Upper floor) 、後者を$L_F$ (Lower floor)と呼ぶ。
$t=0$に最初の砂粒が落ち始める時刻に、上の床は全部で$N_\text{p}$本の砂柱(sand pile)を支えている。
各砂柱には固有の整数$p$ ($0\le p < N_\text{p}$) を割り当てる。
一本の砂柱は$N_\text{s}$個の砂粒から構成される。
砂粒には固有の番号$i$ ($0\le i < N_\text{s}$)をつける。
$0$番目の砂粒が一番下で、$N_\text{s}-1$番目の砂粒が一番上である。
砂粒が落下し、衝突しても、各砂粒の上下の位置関係が変わることはない。


シミュレーション開始時に、すべての砂柱を$U_F$よりも上方に配置し、
$t=0$から全部で$N_0 = N_\text{p} N_\text{s}$個の砂粒の運動方程式を解く。

ある一つの砂柱の中の$i$番目の砂粒の$y$座標を$y_i$と書くと、その運動方程式は、
%==========
\begin{equation} \label{250919171805} 
   m \frac{\mathrm{d}^2 y_i}{\mathrm{d} t^2} = -mg + F_{i,{U_F}} + F_{i,{L_F}}  - F_{i,i+1} + F_{i,i-1}
\end{equation}
%==========
である。
ここで $F_{i,{U_F}}$, $F_{i,{L_F}}$, $F_{i,i+1}$, $F_{i,i-1}$は、
それぞれ上の床$U_F$から受ける抗力、下の床$U_F$から受ける抗力、
上の砂粒と接触した場合に受ける下向きの力(バネ力とダンパー力の和)、
下の砂粒と接触した場合に受ける上向きの力である。


上の床 $U_F$と接触している粒子は常に0個または1個である。
粒子$i$が$U_F$と接触していなければ$F_{i,{U_F}}=0$である。
下の床$L_F$についても同様である。


各砂柱は、上の床$U_F$と接触する可能性のある粒子番号を保持している。
シミュレーション開始時にはその番号は$i=0$(砂柱の中で一番下に位置する砂粒)である。
シミュレーション開始時に$i=0$の砂粒が$U_F$から半径$a$よりも大きな距離だけ離れている、つまり接触していない場合でも、
時間の経過と共にこの粒子0は落下し、いずれ$U_F$と接触する。
この粒子0は$U_F$から上向きの力$F_{i,{U_F}}$の力を受けて減速し、やがて静止するが、
静止する前、あるいはまだ動いている間に、上方から別の粒子$i=1$が落下し、衝突して下向きの力$-F_{01}$を粒子0に及ぼす。
そしてこの粒子1もまたその上の粒子2によって上から押さえつけられる。


十分に時間が経過すると、この砂柱の粒子はすべて静止する。
この状態では、一番下の粒子0は自分自身の重みに加えて、上に積まれた$(N_\text{s}-1)$個の粒子の重みによって下方に押さえつけられ、
その力は床$U_F$とのバネ力$F_{0,{U_F}}$とつり合っている。
つまり$F_{0,{U_F}}=N_\text{s} mg$であり、床$U_F$が砂柱を支える抗力がこの$F_{0,{U_F}}$である。

% . . . . . . . . . . . . . . . . . . . . . . . . . . . . . . . . . . . . . . . . 
      \marginnote{図\ref{250921063433}}
% . . . . . . . . . . . . . . . . . . . . . . . . . . . . . . . . . . . . . . . . 
オリフィスを通過して落下する砂粒の数が時間的に一定というであることが砂時計の基本性質(Beverloo則)である。
このを本研究では次の方法で実装した。
一本の砂柱の砂粒には0番から順番に上に向かって番号がつけてあるので、
当初は粒子0は上方で粒子1と、下方で床面と相互作用する(図\ref{250921063433}左)。
床面が粒子0を支えることで、全ての粒子が床面の上に直線上に積み重なった状態で静止する。
あらかじめ指定した一定の時間$\mu^{-1}$が経過すると、
床面と相互作用する粒子が1に切り替わる。
その結果、粒子0は(床面をすり抜けて)下方に落下する(図\ref{250921063433}中)。
粒子1は床面と衝突するとダンパーによる緩和を経て静止する(図\ref{250921063433}右)。
以下同様に$\mu^{-1}$の時間が経過するごとに床面と相互作用する粒子の番号が切り替わる。
ここでは$y=0$にある床面$U_F$について説明したが、$y=-H_0$にある下の床面$L_F$についても同様に粒子0は$L_F$と衝突し、バネ・ダンパーモデルによる緩和を経て静止する。
下の床面$L_F$は常に粒子0と相互作用するので、粒子0が(したがって全ての砂が)さらに下に落ちることはない。


% . . . . . . . . . . . . . . . . . . . . . . . . . . . . . . . . . . . . . . . . 
      \marginnote{図\ref{250919165122}}
% . . . . . . . . . . . . . . . . . . . . . . . . . . . . . . . . . . . . . . . . 
床$U_F$が相互作用する相手の粒子番号を周期$\mu^{-1}$で切り替えるわけであるが、
$N_P$本ある砂柱でその切り替えの位相をずらすことで砂粒の落下をバラバラにする。
位相をずらす単位は$\Delta \psi = 2\pi/N_P$とした。
$p$番目の砂柱の位相を$p\Delta \psi $とすると、水平方向に並んだ砂柱から順番に次々と砂が落下していくので見た目は綺麗ではあるものの、砂時計というには違和感があるので、それぞれの砂柱にはランダムに$N$個の位相のずれを割り当てた。
図\ref{250919165122}は$N_P=14$で各砂柱中の粒子数$N_S=20$のときのシミュレーションのスナップショットである。



%------------------------------------------
\subsection{シミュレーション結果}\label{250923145905}
%------------------------------------------


% . . . . . . . . . . . . . . . . . . . . . . . . . . . . . . . . . . . . . . . . 
      \marginnote{図\ref{250921210605}}
% . . . . . . . . . . . . . . . . . . . . . . . . . . . . . . . . . . . . . . . . 
図\ref{250921210605}に$N_P=24$, $N_S=40$の結果を示す。
グラフの横軸は時間、縦軸は砂を支える抗力$F$を重さで表した値($F/g$)である。
$N_0=N_P\times N_S=960$, 
$H_0=2.0$~\si{(m)},
$K_0=1.0$~\si{(m)},
$M_0=1.0$~\si{(kg)},
$T_0=10.0$~\si{(s)}とした。
$t=-2~\si{(s)}$にシミュレーションを開始し、
$t=0$に砂の落下を開始する。
時間刻み幅は $\Delta t = 1.996\times 10^{-6}$~\si{(s)} である。


抗力の時間変動が大きいので、
左のグラフでは0.2~\si{(s)}毎の平均値をとっている。% 正確には 1万ステップ毎の平均値


砂の質量$M_0$による重力$M_0$が抗力$F$によって支えらるので、
左のグラフの$t<0$の測定値は$W=F/g=1$の線に乗っている。
$t=0$に砂の落下が始まると砂時計の重さは変動し、最初の砂粒が$y=-H_0$の下の床面に$L_F$に到達するまでは総じて軽くなる。
その後、$t\sim 2$から$t=T_0=10$まではやはり変動が大きいものの、平均をとれば$W=1$よりも大きくなる。
$t=10$以降の短い時間、落下する砂粒による力積の効果により$W$はさらに強くなる。


シミュレーションで得られた砂時計の重さの変化$\Delta W$をみるために、
右のグラフでは0.49~\si{(s)}毎の平均値をとり、縦軸を拡大してプロットした。 % 正確には24万ステップ毎の平均値
式\eqref{250703173052}による理論値は
$\Delta W=2M_0 L_0/T_0^2 = 2.039\times 10^{-3}$~\si{(kg)}である。
シミュレーションで得られた重さの増加は理論とよく一致していることがわかる。


%=============================================
\section{まとめ}
%=============================================

砂が落ちている間、砂時計はわずかに重くなる。 
このよく知られている結果を砂が鉛直方向のみに運動するという1次元モデルに基づく運動方程式を用いて簡便に導出すると同時に、
ソフトスフィアモデルに基づくシミュレーションで定量的に再現した。


砂が落ちるにつれて上の砂層の質量$M_1$と厚さ$K_1$は時間の線形で減少し、
下の砂層の質量$M_2$と厚さ$K_2$は線形に上昇する。
$+y$を鉛直上方にとると
$\dot{M}_1<0$, $\dot{K}_1<0$であるが、上の砂層の運動量$P_1$の時間微分$\dot{P}_1=\dot{M}_1 \dot{K}_1 = M_0K_0/2T_0^2>0$は上向きである。
同様に下の砂層の運動量の時間微分$\dot{P}_2=\dot{M}_2 \dot{K}_2 = M_0K_0/2T_0^2 >0$も上向きである。


この二つに加えて見逃し易いが決して無視できない大きさを持つ効果は、
下の層が厚くなるにつれて、砂の落下距離が短くなるためことから、落下中の砂がもつ運動量$P_3<0$の時間微分の効果$\dot{P}_3=M_0K_0/T_0^2>0$である。
この3つの力を合わせると式\eqref{250703173105}で示した力の増分、
つまり砂時計の重さの起源はこの3つの効果の和による力$\Delta F=2M_0K_0/T_0^2$である。
手のひらに乗るほどの砂時計を想定し、砂が落ちる前の砂の層の厚さ$K_0$を$\mathcal{O}(10^{-1})$~\si{(m)}、
砂時計の計時時間$T_0=\mathcal{O(10^{2})}$とすると、
砂の質量$M_0$にかかる重力$M_0 g$と比較した$\Delta F$の相対的な大きさは$2K_0/gT_0^2=\mathcal{O}(10^{-5})$と極めて小さいので、手に持った人が感じられる差ではない。


このようなわずかな変化をコンピュータシミュレーションで定量的に再現するために、
我々は単純なバネ=ダンパーモデル(ソフトスフィアモデル)に基づく粒子法シミュレーションを行った。
複数の粒子が同時に一つの粒子に接触するような状況も解けるのがソフトスフィアモデルの利点である。
砂時計の粒子群の運動を精度良く追跡するためにはソフトスフィアモデルが相応しい。


このシミュレーションの難しさは、シミュレーションで扱うべきマクロな時間スケールとミクロな時間スケールが大きく異なることにある。
ミクロな時間スケールは、粒子の衝突時間スケールである。
自由落下速度$v_f$で落下してきた半径$a$の砂粒が、下の砂粒に衝突したことを検知するためには、
シミュレーションの時間刻み幅$\Delta t$が$\Delta t\ll a/v_f$でなければならない。
第\ref{250923145905}章で紹介したシミュレーションの場合、$\Delta t = 1.996\times 10^{-6}$~\si{(s)} とした。
一方、マクロな時間スケールは砂時計の計時時間$T_0$である。
計時時間$T_0$を$\mathcal{O}(10^{1})$と比較的小さく設定したとしても、
シミュレーションの時間積分ステップ数は少なくとも$T_0 / \Delta t=5\times 10^6$は必要である。


多数の粒子群の3次元的な運動方程式を500万ステップ以上にわたって数値積分するのは難しいが、
このシミュレーションでは、鉛直方向の運動だけを考慮する1次元的なモデルを採用することでこの困難を克服した。
1次元の粒子モデルでは、一つの粒子は最大、上下二つの隣の粒子とだけ接触し、しかも接触する相手は変わらないので、
取り扱いが単純で高速である。


本研究で開発したシミュレーションプログラムはProcessing言語で書いた。
この言語では、浮動小数点数変数の種類は単精度実数しかデフォルトで用意されていないが、
本研究では倍精度浮動小数点数で計算した。
%シミュレーションプログラムはGitHubで公開する。


\bibliographystyle{amsplain} 
\bibliography{hourglass_weight}

\appendix

%=============================================
\section{下の砂層の厚さの時間変化}\label{250919092352}
%=============================================

落下中の砂の総数は
%==========
\begin{equation} \label{250917175724} 
   N_3(t) = \mu \tau_f(t) = \mu \sqrt{\frac{2K_3(t)}{g}} = \mu \sqrt{\frac{2(H_0-K_2(t))}{g}} = \mu \sqrt{\frac{2}{g}}\sqrt{H_0-K_2(t)}
\end{equation}
%==========
である。したがってその質量は
%==========
\begin{equation} \label{250917175839} 
   M_3(t) = mN_3(t) %= m \mu \sqrt{\frac{2(H_0-K_2(t))}{g}} 
   		 = m\mu\sqrt{\frac{2}{g}}\sqrt{H_0-K_2(t)}
\end{equation}
%==========
である。




砂の総質量は
%==========
\begin{equation} \label{250917175922} 
   M_0 = M_1(t) + M_2(t) + M_3(t)
\end{equation}
%==========
は定数である。
%これは
%%==========
%\begin{equation} \label{250919085544} 
%   \sigma K_0 =  \sigma K_1(t) + \sigma K_2(t) + m\mu\sqrt{\frac{2}{g}}\sqrt{H_0-K_2(t)}
%\end{equation}
%%==========
%==========
\begin{equation} \label{250920140526}
   K_0 =  K_1(t) + K_2(t) + \frac{m\mu}{\sigma}\sqrt{\frac{2}{g}}\sqrt{H_0-K_2(t)}
\end{equation}
%==========
この式の時間微分をとる。
%==========
\begin{equation} \label{250920140735} 
   0 = \dot{K}_1(t) +   \dot{K}_2(t)  + \frac{m\mu}{\sigma}\sqrt{\frac{2}{g}} \frac{\mathrm{d} }{\mathrm{d} t}\sqrt{H_0-K_2(t)} 
\end{equation}
%==========
式\eqref{250919084914}より
%==========
\begin{equation} \label{250919085137} 
    0 =  -\frac{m\mu}{\sigma} +   \dot{K}_2(t)  + \frac{m\mu}{\sigma}\sqrt{\frac{2}{g}} \frac{\mathrm{d} }{\mathrm{d} t}\sqrt{H_0-K_2(t)}
\end{equation}
%==========
両辺を$H_0$で割ると
%==========
\begin{equation} \label{250920140957} 
    0 =  -\frac{m\mu}{\sigma H_0} +   \frac{\dot{K}_2(t)}{H_0}  + \frac{m\mu}{\sigma}\sqrt{\frac{2}{gH_0}} \frac{\mathrm{d} }{\mathrm{d} t}\sqrt{1-\frac{K_2(t)}{H_0}}
\end{equation}
%==========
ここで$K_2(t)$を$H_0$で規格化し
%==========
\begin{equation} \label{250920141535} 
   k_2(t) = \frac{K_2(t)}{H_0}
\end{equation}
%==========
を導入する。
さらに時間の逆数の次元をもつ定数
%==========
\begin{equation} \label{250920141440} 
   \beta = \frac{m\mu}{\sigma H_0}
\end{equation}
%==========
と無次元定数
%==========
\begin{equation} \label{250920141156} 
   \epsilon = \frac{m\mu}{\sigma}\sqrt{\frac{2}{gH_0}} = 2\frac{K_0/T_0}{v_{f0}}
\end{equation}
%==========
を定義する。
右辺最終項の分子は砂時計の速度スケール(砂層の初期高さと計時時間の比)で、分母は砂粒の(最大)自由落下速度である。
したがって$\epsilon$は微少量である。
%==========
\begin{equation} \label{250917183422} 
    \epsilon \ll 1
\end{equation}
%==========
あとで$\mathcal{O}(\epsilon^2)$を無視する近似を行う。
すると$k_2(t)$に対する微分方程式
%==========
\begin{equation} \label{250920141635} 
    \beta =  \dot{k}_2(t)  + \epsilon \frac{\mathrm{d} }{\mathrm{d} t}\sqrt{1-k_2(t)}
\end{equation}
%==========
を得る。
これを解くために変数変換をする
%==========
\begin{equation} \label{250920142319} 
   \zeta(t) =  \sqrt{1-k_2(t)}
\end{equation}
%==========
つまり
%==========
\begin{equation} \label{250920142334} 
   k_2(t) = 1-\zeta^2(t)
\end{equation}
%==========
である。
解くべき微分方程式は
%==========
\begin{equation} \label{250920142453} 
   \beta = -2\zeta\dot{\zeta} + \epsilon\dot{\zeta}
\end{equation}
%==========
つまり
%==========
\begin{equation} \label{250920142635} 
   (2\zeta-\epsilon)\dot{\zeta} = -\beta
\end{equation}
%==========
である。
これを解く。
%==========
\begin{equation} \label{250920142711} 
   (2\zeta-\epsilon)\,\mathrm{d}\zeta = -\beta\,\mathrm{d} t
\end{equation}
%==========
%==========
\begin{equation} \label{250920142804} 
    \int (2\zeta-\epsilon)\,\mathrm{d}\zeta = -\int \beta\,\mathrm{d} t 
\end{equation}
%==========
%==========
\begin{equation} \label{250920142835} 
   \zeta^2 - \epsilon \zeta = -\beta t + C
\end{equation}
%==========
$K_2$(つまり$\zeta$)は$t\ge \tau_{f0}$から変化し始めるので、
初期条件として$t=\tau_{f0}$をとるべきだが、今は簡単のため時刻の原点をずらして$t=0$での初期条件を$\zeta=1$(つまり$K_2=0$)とした解を求め、
最後に時刻を平行移動して元に戻す。
初期条件$\zeta(0)=1$より積分定数が$C=1-\epsilon$と決まるので、
%==========
\begin{equation} \label{250920143258} 
   \zeta^2 - \epsilon \zeta + \beta t =  1 - \epsilon
\end{equation}
%==========
%==========
\begin{equation} \label{250920143924} 
  \zeta^2 - \epsilon \zeta + \beta t +\epsilon -  1  = 0
\end{equation}
%==========
この2次方程式を解く
%==========
\begin{equation} \label{250920144013} 
   \zeta = \frac{1}{2}
   			\left(
				\epsilon \pm \sqrt{\epsilon^2-4(\beta t+\epsilon-1)}
			\right)   			
\end{equation}
%==========
%==========
\begin{equation} \label{250920144311} 
   \zeta = \frac{1}{2}
   			\left(
				\epsilon \pm \sqrt{\epsilon^2-4\beta t-4\epsilon+4)}
			\right)   		
\end{equation}
%==========
%==========
\begin{equation} \label{250920144425} 
   \zeta = \frac{1}{2}
   			\left(
				\epsilon \pm \sqrt{(2-\epsilon)^2-4\beta t}
			\right)   	
\end{equation}
%==========
初期条件から複合は+をとる。
%==========
\begin{equation} \label{250920144626} 
   \zeta = \frac{1}{2}
   			\left(
				\epsilon + \sqrt{(2-\epsilon)^2-4\beta t}
			\right)   	
\end{equation}
%==========
式\eqref{250920142334}より
%==========
\begin{equation} \label{250920144518} 
  k_2  = 1-\frac{1}{4}
   			\left(
				\epsilon + \sqrt{(2-\epsilon)^2-4\beta t}
			\right)^2
\end{equation}
%==========
これから$\mathcal{O}(\epsilon^2)$を無視する近似を行う。
$\epsilon$は上の式の右辺、最後の項の中だけにあり、しかもこの項は括弧の中が2乗されているので、
今の近似のためにはこの括弧の中の$\epsilon$の1次までとればよい。
%==========
\begin{equation} \label{250920145208} 
  k_2  = 1-\frac{1}{4}
   			\left(
				\epsilon + \sqrt{4-4\epsilon-4\beta t}
			\right)^2 + \mathcal{O}(\epsilon^2)
\end{equation}
%==========
%==========
\begin{equation} \label{250920145322} 
  k_2  = 1-\frac{1}{4}
   			\left(
				\epsilon + 2\sqrt{1-\epsilon-\beta t}
			\right)^2+ \mathcal{O}(\epsilon^2)
\end{equation}
%==========
%==========
\begin{equation} \label{250920145438} 
  k_2  = 1-\frac{1}{4}
   			\left(
				\epsilon + 2\sqrt{1-\beta t-\epsilon}
			\right)^2+ \mathcal{O}(\epsilon^2)
\end{equation}
%==========
%==========
\begin{equation} \label{250920145507} 
  k_2  = 1-\frac{1}{4}
   			\left(
				\epsilon + 2\sqrt{(1-\beta t)(1-\frac{\epsilon}{1-\beta t})}
			\right)^2+ \mathcal{O}(\epsilon^2)
\end{equation}
%==========
%==========
\begin{equation} \label{250920145549} 
  k_2  = 1-\frac{1}{4}
   			\left(
				\epsilon + 2\sqrt{1-\beta t}\sqrt{1-\frac{\epsilon}{1-\beta t}}
			\right)^2+ \mathcal{O}(\epsilon^2)
\end{equation}
%==========
%==========
\begin{equation} \label{250920145621} 
  k_2  = 1-\frac{1}{4}
   			\left(
				\epsilon + 2\sqrt{1-\beta t}\left(1-\frac{\epsilon}{2(1-\beta t)}\right)
			\right)^2+ \mathcal{O}(\epsilon^2)
\end{equation}
%==========
%==========
\begin{equation} \label{250920150047} 
  k_2  = 1-\frac{1}{4}
   			\left(
				\epsilon + 2\sqrt{1-\beta t}-\frac{\epsilon}{\sqrt{1-\beta t}}
			\right)^2+ \mathcal{O}(\epsilon^2)
\end{equation}
%==========
%==========
\begin{equation} \label{250920150330} 
  k_2  = 1-\frac{1}{4}
   			\left(
				 2\sqrt{1-\beta t} + \epsilon -\frac{\epsilon}{\sqrt{1-\beta t}}
			\right)^2+ \mathcal{O}(\epsilon^2)
\end{equation}
%==========
%==========
\begin{equation} \label{250920150359} 
  k_2  = 1-\frac{1}{4}
   			\left(
				 2\sqrt{1-\beta t} \left(
				 		   1
				 			+ \frac{\epsilon}{2\sqrt{1-\beta t}} -\frac{\epsilon}{2(1-\beta t)}
								\right)
			\right)^2+ \mathcal{O}(\epsilon^2)
\end{equation}
%==========
%==========
\begin{equation} \label{250920150657} 
  k_2  = 1-\frac{1}{4}
  			 4(1-\beta t)
			 \left(
				 	  1
				 			+ \frac{\epsilon}{\sqrt{1-\beta t}} -\frac{\epsilon}{(1-\beta t)}
			\right)+ \mathcal{O}(\epsilon^2)
\end{equation}
%==========
%==========
\begin{equation} \label{250920150941} 
  k_2  = 1- (1-\beta t)
			 \left(
				 	  1
				 			+ \frac{\epsilon}{\sqrt{1-\beta t}} -\frac{\epsilon}{(1-\beta t)}
			\right)+ \mathcal{O}(\epsilon^2)
\end{equation}
%==========
%==========
\begin{equation} \label{250920151020} 
  k_2  = 1- (1-\beta t)
					- \epsilon\sqrt{1-\beta t} +\epsilon
			+ \mathcal{O}(\epsilon^2)
\end{equation}
%==========
%==========
\begin{equation} \label{250920151147} 
  k_2  = \beta t +  \epsilon\left(1-\sqrt{1-\beta t}\right)
			+ \mathcal{O}(\epsilon^2)
\end{equation}
%==========
%初期条件の設定時刻を$t=0$から$t=\tau_{f0}$にずらすと
%%==========
%\begin{equation} \label{250920163140} 
%    k_2  = \beta (t-\tau_{f0}) +  \epsilon\left(\sqrt{1-\beta (t-\tau_{f0})} -1\right)
%			+ \mathcal{O}(\epsilon^2)  
%\end{equation}
%%==========
時間微分は
%==========
\begin{equation} \label{250920152117} 
   \dot{k}_2 = \beta + \frac{\beta \epsilon}{2\sqrt{1-\beta t}}+ \mathcal{O}(\epsilon^2)
\end{equation}
%==========
である。
$K_2$にもどすと、
%==========
\begin{equation} \label{250920163650}
  K_2  = \frac{m\mu}{\sigma} t +  \epsilon H_0 \left(1-\sqrt{1-\beta t} \right)
			+ \mathcal{O}(\epsilon^2)
\end{equation}
%==========
%==========
\begin{equation} \label{250920161340} 
   \dot{K}_2  = \frac{m\mu}{\sigma}\left( 1+ \frac{\epsilon}{2\sqrt{1-\beta t}}\right)+ \mathcal{O}(\epsilon^2)
\end{equation}
%==========






以上の解は微分方程式を$t=0$で$K_2=0$($\zeta=1$)という初期条件で解いたものであったが、本来は
$t=\tau_{f0}$で$K_2=0$とすべきであった。
この初期条件の下での解は
%==========
\begin{equation} \label{250920170419}
  K_2  = \frac{m\mu}{\sigma} (t-\tau_{f0}) +  \epsilon H_0 \left(1-\sqrt{1-\beta  (t-\tau_{f0})} \right)
			+ \mathcal{O}(\epsilon^2)
\end{equation}
%==========
%==========
\begin{equation} \label{250920170423}
   \dot{K}_2  = \frac{m\mu}{\sigma}\left( 1+ \frac{\epsilon}{2\sqrt{1-\beta  (t-\tau_{f0})}}\right)+ \mathcal{O}(\epsilon^2)
\end{equation}
%==========
である。



\newpage
%=============================================
\section*{図表一覧}
%=============================================
\begin{description}
  % fig01
  \item[図\ref{250919103231}]
    砂時計内部の砂を支える抗力と本研究における砂のモデル。
    砂は質量が変化する物体と考える。
    落下前の上の砂層を物体1、落下後の下の砂層を物体2、落下中の砂粒群を物体3とする。
    砂時計の重さが変わるのは、砂時計の上の「床面」が物体1を支える抗力$F_1$と、
    下の「床面」の抗力$F_2$の和が時間変化するためである。

  % fig02
  \item[図\ref{250921122102}]
    砂粒の落下テスト。
    半径$a=1\times 10^{-2}$~\si{(m)}の砂粒を、高さ$y=1~\si{(m)}$から自由落下させる。
    $y=0$に設定した床面に非弾性衝突して静止する。
    横軸は時間、縦軸は粒子の中心の$y$座標。
    右のグラフは左のグラフを衝突点付近で拡大したもの。
    $y=a$で床面に接触し、粒子のバネとダンパーによって衝突のエネルギーが吸収される。

  % fig03
  \item[図\ref{250921063433}]
    オリフィスのモデル。
    時間の経過と共に一定の割合$\mu~\si{(s^{-1})}$で砂粒が落下する粉体特有の現象をソフトスフィアモデルに組み込むために、
    本シミュレーションでは床面のと接触(抗力)を担う粒子を一定の割合で切り替える。
    左の図では粒子0の下側のバネ(マゼンタ色)が床面と接触し、抗力を受けている。
    $1/\mu$~\si{(s)}の時間が経過すると、床面と相互作用する粒子は1に切り替わる(真ん中の図)。
    粒子0は床に支えられていないので、下方に落下する。
    粒子1(およびそれよりも上の粒子2等)は重力によって落下するが、
    すぐに床面に衝突して次の切り替え時刻まで、床面に支えられる(右の図)。

  % fig04
  \item[図\ref{250919165122}] 
    砂柱の数$N_P=14$で、各砂柱中の粒子数$N_S=20$でのシミュレーションのスナップショット。
    $U_F$は上の床面、$L_F$は下の床面を示す。
    砂が落ちる前(左の図)、
    落ちている最中(真ん中の図)、
    全ての砂が落ちた後(右の図)。

  % fig05
  \item[図\ref{250921210605}] 
    質量$M_0=1$~\si{(kg)}の砂を支える抗力$F$の時間変化を重さ$W=F/g$で示したグラフ。
    左のグラフは時間間隔$0.2$~\si{(s)}で平均をとった。
    砂の落下は$t=0$から$t=10$~\si{(s)}までである。
    砂の落下前後、砂が静止している状態では$W=1$の線上にデータが乗っている。
    右のグラフは時間間隔$0.49$~\si{(s)}で平均をとり、縦軸を拡大してプロットしたもの。
    砂の重さの増分$\Delta W$の理論値は$2.039\times10^{-3}$~\si{(kg)}は``theory''とラベルをつけた水平線で示した。
\end{description}



\newpage
%=========================================================================
\section*{著者紹介}
%=========================================================================

\noindent
陰山 聡(会員)

神戸大学システム情報学研究科教授。
広島大学大学院理学研究科修了。
核融合科学研究所理論・シミュレーション研究センター、海洋研究開発機構地球シミュレータセンター、神戸大学工学研究科を経て2010年より現職。
主に地磁気に関連した計算科学と可視化を含めたシミュレーション手法の研究を行っている。



\vspace{1em}
\noindent
中島 涼輔(非会員)

XXXXXXXXXXXXXXXXXXXXXXXXXXXX。

XXXXXXXXXXXXXXXXXXXXXXXXXXXX。

\vspace{1em}
\noindent
中戸 昂明(非会員)

神戸大学システム情報学部学生。



\newpage
%-------------------------------------------------------
\begin{figure}[ht]   \centering   
  \includegraphics[%
     height=1.0\textheight,%
       width=0.6\hsize,keepaspectratio]%
         {./figs/schematic_objects_and_forces.png}   
      \caption{砂時計内部の砂を支える抗力と本研究における砂のモデル。
      砂は質量が変化する物体と考える。
      落下前の上の砂層を物体1、落下後の下の砂層を物体2、落下中の砂粒群を物体3とする。
      砂時計の重さが変わるのは、砂時計の上の「床面」が物体1を支える抗力$F_1$と、
      下の「床面」の抗力$F_2$の和が時間変化するためである。}
      \label{250919103231}
\end{figure} 
%-------------------------------------------------------

\newpage
%-------------------------------------------------------
\begin{figure}[ht]   \centering   
  \includegraphics[%
     height=1.0\textheight,%
       width=1.0\hsize,keepaspectratio]%
         {./figs/fallingSingleGrain.png}   
      \caption{砂粒の落下テスト。
      半径$a=1\times 10^{-2}$~\si{(m)}の砂粒を、高さ$y=1~\si{(m)}$から自由落下させる。
      $y=0$に設定した床面に非弾性衝突して静止する。
      横軸は時間、縦軸は粒子の中心の$y$座標。
      右のグラフは左のグラフを衝突点付近で拡大したもの。
      $y=a$で床面に接触し、粒子のバネとダンパーによって衝突のエネルギーが吸収される。}
      \label{250921122102}
\end{figure} 
%-------------------------------------------------------

\newpage
%-------------------------------------------------------
\begin{figure}[ht]   \centering   
  \includegraphics[%
     height=1.0\textheight,%
       width=0.5\hsize,keepaspectratio]%
         {./figs/schematic_orifice}   
      \caption{オリフィスのモデル。
      時間の経過と共に一定の割合$\mu~\si{(s^{-1})}$で砂粒が落下する粉体特有の現象をソフトスフィアモデルに組み込むために、
      本シミュレーションでは床面のと接触(抗力)を担う粒子を一定の割合で切り替える。
      左の図では粒子0の下側のバネ(マゼンタ色)が床面と接触し、抗力を受けている。
      $1/\mu$~\si{(s)}の時間が経過すると、床面と相互作用する粒子は1に切り替わる(真ん中の図)。
      粒子0は床に支えられていないので、下方に落下する。
      粒子1(およびそれよりも上の粒子2等)は重力によって落下するが、
      すぐに床面に衝突して次の切り替え時刻まで、床面に支えられる(右の図)。
      }
      \label{250921063433}
\end{figure} 
%-------------------------------------------------------

\newpage
%-------------------------------------------------------
\begin{figure}[ht]   \centering   
  \includegraphics[%
     height=1.0\textheight,%
       width=1.0\hsize,keepaspectratio]%
         {./figs/fallingSands.jpg}   
      \caption{砂柱の数$N_P=14$で、各砂柱中の粒子数$N_S=20$でのシミュレーションのスナップショット。
      $U_F$は上の床面、$L_F$は下の床面を示す。
      砂が落ちる前(左の図)、
      落ちている最中(真ん中の図)、
      全ての砂が落ちた後(右の図)。
      }
      \label{250919165122}
\end{figure} 
%-------------------------------------------------------

\newpage
%-------------------------------------------------------
\begin{figure}[ht]   \centering   
  \includegraphics[%
     height=1.0\textheight,%
       width=1.0\hsize,keepaspectratio]%
         {./figs/sandWeight.png}   
      \caption{質量$M_0=1$~\si{(kg)}の砂を支える抗力$F$の時間変化を重さ$W=F/g$で示したグラフ。
      左のグラフは時間間隔$0.2$~\si{(s)}で平均をとった。
      砂の落下は$t=0$から$t=10$~\si{(s)}までである。
      砂の落下前後、砂が静止している状態では$W=1$の線上にデータが乗っている。
      右のグラフは時間間隔$0.49$~\si{(s)}で平均をとり、縦軸を拡大してプロットしたもの。
      砂の重さの増分$\Delta W$の理論値は$2.039\times10^{-3}$~\si{(kg)}は``theory''とラベルをつけた水平線で示した。
      }
      \label{250921210605}
\end{figure} 
%-------------------------------------------------------

\end{document}
