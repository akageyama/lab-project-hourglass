\documentclass[]{article}
\usepackage{siunitx}
\usepackage[dvipdfm]{hyperref}
\usepackage[dvipdfmx]{graphicx}
\usepackage{bm}
\usepackage{fancyhdr}
\usepackage{indentfirst}
\usepackage{listings}
\usepackage{amsmath,amssymb}
\usepackage{here}
\usepackage{ascmac}
\usepackage[dvipsnames]{xcolor}
\usepackage{url}
\usepackage{colortbl}
\usepackage{comment}

\title{砂時計のシミュレーション}

\author{中戸 昂明$^\dagger$, 中島 涼輔$^\ddagger$, 陰山 聡$^\dagger$\\[0.5em] 
        $^\dagger$神戸大学システム情報学部, $^\ddagger$神戸大学システム情報学研究科}
\date{\today}

\begin{comment}
\end{comment}

\begin{document}

\maketitle


\begin{abstract}
砂時計の砂が落ちている間、砂時計の重さはわずかに重くなることが知られている。
その重さの変化はとても小さいので、この効果の実験的な検証は容易ではない。
ソフトスフィアモデルに基づいた個別要素法シミュレーションにより、砂時計の重さの変化を再現した。
\end{abstract}


%=============================================
\section{はじめに}
%=============================================
砂時計の砂が落ち始める前の砂時計全体の重さと、
砂がすべて落ちきった後の砂時計全体の重さはもちろん変わらない。
では、砂が落ちている最中の砂時計の重さはどうであろうか。
これは大学で学ぶ力学、特に力積の概念を使うよい練習問題になりそうであるが、
実際には力学の教科書などで取り上げられる例は少ない。
それはこの問題の正解が「理論上は少しだけ重くなるが、実際上は変わらないといってよい」というあまりすっきりしないものであるために初学者にはあまり教育的でないからかもしれない。


砂が落ちている間、砂時計がわずかに重くなるであろうことは砂の重心の運動から容易に予想される。
計時開始前には砂時計の上部にあった砂粒全体の重心は、
計時後には下部に移動している。
上部の砂層の厚さとそこに含まれる砂の質量は時間$t$の線形で減少する一方、
下部の砂層の厚さと質量は線形で増加する。
したがって全体の重心位置は$t$の2次関数で減少(落下)するが、その2階微分(加速度)の向きは自由落下とは反対に上向きである。
つまり砂時計全体は重くなるはずである。


ただし、重心移動の距離はわずかで最大でも砂時計の高さ程度あり、
しかもかなり時間をかけたゆっくりとした移動(砂時計が3分計であれば3分間)なので、
その加速度は必然的に小さく、したがって砂時計全体の重さに与える影響は小さいことは予想できる。


GardnerがScientific American誌で水中を浮力で浮かぶ砂時計を使ったおもちゃを紹介したときが砂時計の重さが大きな注目を集めた最初であろう\cite{Gardner1966-eu}。
その後すぐに、落下した砂が砂時計の底面に与える力積(つまり砂時計を重くする効果)が、
落下中の砂の分だけ砂時計の重さが軽くなる効果とちょうどキャンセルするという指摘がされた\cite{Reid1967-jq}。
そこでは、力学教育のよい題材にこの問題はなるであろうとも指摘されている。


その後、砂の重心の移動まで考慮に入れると砂時計はわずかに重くなることが指摘された\cite{ShenUnknown-pk}。
著者等は理論解析に加えて実験を行い、ガラスの粉末を使った「砂」時計の重量を測定して、実際に重くなることを示した。
重さの変化は、理論予想が49~\unit{mg}に対して実験結果は$30\sim 80$~\unit{mg}の範囲であった。
おそらくこれがこの効果を最初に実証した実験であろう。


日常目にする砂時計のオリフィスは一つで、落下する砂粒の流れも一本である。
下面に降り積もった砂が勾配をつくることで現象が複雑になり重さ変動の時間揺らぎも生じる。
この問題を解決するために、Tuinstra等は単一のオリフィスではなく多数(230個)の穴を持つ「ふるい」で区切られた円筒状の砂時計モデルを作成し、
(本物の)砂$1.6$~\si{kg}を使った実験を行った\cite{Tuinstra2010-wk}。
理論的的には$0.116$~\si{g}の重さの増加が期待され、測定結果は$0.121$~\si{g}であった。


金属粉末を使った実験で、砂時計の重さの時間変化が詳細に解析されている\cite{Sack2017-rq}。
時間変化は主に3つの時期に分けることができる。
最初期、砂が落ち始めた直後から最初の砂粒が下面に到達するまでの短い間、砂時計の重量は軽くなる。
これは空中にいる砂粒の重量分だけ重みがかからないためである。
また、最終期、最後の砂粒が落ちている短い間は、砂時計は重くなる。
これは下面を叩く砂粒達が与える力積をキャンセルするだけの数の砂が空中にないためである。
この二つ短い遷移時間の間、浮いている砂粒の軽さと床面に衝突する砂粒が与える力積がキャンセルするが、
冒頭に述べた重心に移動効果により、重量が一定の値でわずかに重くなる時期が続く。


砂時計の重さの問題は、多くの非自明な要因が関係する力学の問題として興味深いだけでなく、
質量が時間変化する系の典型例として教育的にも重要であると指摘されている\cite{Kassandrov2023-tn}。


コンピュータシミュレーションでこの効果、つまり砂が落下している最中の砂時計がわずかに重くなる効果を実証した研究はまだないようである。
本研究の目的は、簡単でありながら高精度な砂時計のコンピュータシミュレーションモデルを開発することである。
我々のモデルは、落下中の砂の分だけ軽くなる効果と落下した砂が与える力積がちょうどキャンセルするという基本的な効果を再現することはもちろん、
砂の重心移動によるわずかな重さの変化という微弱な効果も定量的に再現する。
このシミュレーションはプログラミング初心者にも扱えるようデザインされたProcessing言語\footnote{\url{https://processing.org}}で書かれており、
そのソースコードは公開する予定である\footnote{\url{https://github.com/akageyama/lab-project-hourglass}}。
ノートPCで実行可能なほど小規模なシミュレーションなので、
この興味深い力学問題を直接体験することができる。


%=============================================
\section{問題設定}
%=============================================

Tuinstra等の実験ではくびれのない円筒形の容器内部で落下する多数の砂が同時に落下するモデルが用いられた\cite{Tuinstra2010-wk}。
我々も砂の水平方向の運動を無視し、鉛直方向の1次元的な運動を考察する。


$T_0$秒間を測る砂時計を考える。
重力加速度を$g$、砂粒の総数を$N$、砂全体の質量を$M_0$とすると、
砂粒一つの質量$m$は、
%==========
\begin{equation} \label{250517100232} 
   m = \frac{M_0}{N} 
\end{equation}
%==========
である。
1秒間に落ちる砂粒の数を
%==========
\begin{equation} \label{250515082817} 
   \mu = \frac{N}{T_0}
\end{equation}
%==========
とする。


鉛直上向きに$y$軸をとり、
オリフィスの位置を$y=0$とする。
オリフィスから砂時計のガラスの底面までの距離、
つまりガラス内部の空間の高さの半分を$H_0$~(\si{m})とする。
砂が移動するガラス内部の天井面と床面の位置は
それぞれ$y=H_0$と$y=-H_0$である。


オリフィスの上の砂の層の重心の$y$座標を$Y_1$、質量を$M_1$とし、
砂時計の下部に落下した砂の層の重心の$y$座標を$Y_2$、質量を$M_2$とする。
$\dot{Y}_1$は$t$の線形で減少し



一つの砂粒がオリフィスから落下し、下部に積み重なった砂の層の表面まで到達するのにかかる時間、つまり自由落下時間を$\tau_f$とし、
下に落ちた砂粒は跳ね返らずに短い時間で静止すると仮定する。
落下した瞬間の砂粒の速さ(自由落下速度)は
%==========
\begin{equation} \label{250512190520} 
   V_F = g\tau_f
\end{equation}
%==========
なので、落下によって静止した砂粒の運動量の変化は
%==========
\begin{equation} \label{250512190655} 
   p = mg\tau_f
\end{equation}
%==========
である。
$\Delta t$秒間に落下する$\mu \Delta t$個の砂粒が下面に与える運動量の総量$p\mu \Delta t$は力積$F_\mathrm{i} \Delta t$に等しいので、
%%==========
%\begin{equation} \label{250512191540} 
%   F_\mathrm{i} \Delta t = \mu \Delta t  mg\tau_f
%\end{equation}
%%==========
床面に与える平均力は
%==========
\begin{equation} \label{250515093535} 
  F_\mathrm{i}   = \mu m g \tau_f
\end{equation}
%==========
である。


一方、ある瞬間に落下している途中、つまり砂時計のオリフィスから下面までの中空にいる砂粒は$\mu \tau_f$個ある。
一つの砂粒にかかる重力は$mg$なので、
砂時計全体の重さはこの落下中の砂粒の重さ
%==========
\begin{equation} \label{250513083619} 
   F_\mathrm{s} = \mu \tau_f\times mg 
\end{equation}
%==========
だけ軽くなる。
これは力積による平均力、つまり式~\eqref{250515093535}と等しい。
従って砂が落ちている最中の砂時計全体の重さは変わらない、という結論になりそうだが、厳密にはこれは正しくない。


%=============================================
\section{運動方程式に基づいた計算}
%=============================================

砂時計の重心が移動する効果を考慮するためにはこの問題を静力学の問題としてではなく、運動方程式に基づいて考えるべきである。


オリフィス付近を除いた砂時計の断面積を$S_0$~(\si{m^2})、
計時を開始する前の砂の層の厚さを$K_0$~(m)、
砂粒の質量密度を$\rho$~(\si{kg/m^3})とすると
砂の全質量が$M_0$~(\si{kg})なので
%==========
\begin{equation} \label{250703152145} 
   M_0 = N m = S_0 K_0 \rho
\end{equation}
%==========
である。


計時開始時刻を$t=0$とすると、
時刻$t$では$\mu t$個の砂粒がオリフィスから落下しているので、
この時刻にオリフィスの上の部分にある砂、つまりまだ落下していない砂の数は
%==========
\begin{equation} \label{250703152453} 
   N_1(t) =  N - \mu t
\end{equation}
%==========
である。
したがってオリフィス上部の砂の質量は
%==========
\begin{equation} \label{250703152655} 
   M_1(t) = m (N-\mu t)
\end{equation}
%==========
である。
この時刻におけるオリフィス上部の砂層の厚さは
%==========
\begin{equation} \label{250703152748} 
   K_1(t) = \frac{M_1(t)}{\rho S_0} = \frac{m}{\rho S_0}(N-\mu t)
\end{equation}
%==========
である。
この層の重心の$y$座標を$Y_1(t)$とすると
%==========
\begin{equation} \label{250703153013} 
   Y_1(t) = \frac{K_1(t)}{2} = \frac{m}{2\rho S_0} ( N - \mu t)
\end{equation}
%==========



\color{blue}
上の砂層の質量は、その厚さに比例するので
%==========
\begin{equation} \label{250917175554} 
   M_1(t) = \frac{M_0}{K_0} K_1(t)
\end{equation}
%==========
である。
同様に下の砂層の厚さを$K_2(t)$とすると質量は
%==========
\begin{equation} \label{250917175633} 
   M_2(t) =  \frac{M_0}{K_0} K_2(t)
\end{equation}
%==========
である。
時刻$t=0$に落下した最初の砂粒が下の床に到達するまでの時間、つまり自由落下時間を
%==========
\begin{equation} \label{250703153227} 
   \tau_{f0} = \sqrt{\frac{2H_0}{g}}
\end{equation}
%==========
とすると、
$M_2(t)$が正になるのは$t=\tau_{f0}$以降である。


落下中の砂粒の総質量を求める。
時刻$t$に落下中の砂粒の数は
%==========
\begin{equation} \label{250917175724} 
   N_3(t) = \mu \tau_f(t) = \mu \sqrt{\frac{2(H_0-K_2(t))}{g}} 
\end{equation}
%==========
であるから、
%==========
\begin{equation} \label{250917175839} 
   M_3(t) = m \mu \sqrt{\frac{2}{g}} \sqrt{H_0-K_2(t)}
\end{equation}
%==========
である。


質量保存則より
%==========
\begin{equation} \label{250917175922} 
   M_0 = M_1(t) + M_2(t) + M_3(t)
\end{equation}
%==========
時間微分をとると、
%==========
\begin{equation} \label{250917180035} 
   -M_0 \dot{K}_1(t) = M_0 \dot{K}_2(t)  - \frac{m\mu K_0}{\sqrt{2g}} \frac{\dot{K}_2(t)}{\sqrt{H_0-K_2(t)}}
\end{equation}
%==========
%==========
\begin{equation} \label{250917180138} 
   \dot{K}_1 = -\frac{m\mu}{\rho S_0}
\end{equation}
%==========
と
%==========
\begin{equation} \label{250917180220} 
   M_0 =  S_0 K_0 \rho
\end{equation}
%==========
を代入すると
%==========
\begin{equation} \label{250917180242} 
   \frac{K_0 m \mu}{M_0} =  \left(1 - \frac{1}{\sqrt{2g}}\frac{m \mu K_0}{M_0}  \frac{1}{\sqrt{H_0-K_2(t)}}\right) \dot{K}_2(t)
\end{equation}
%==========
ここで定数
%==========
\begin{equation} \label{250917180407} 
   A = \frac{m \mu K_0}{M_0}
\end{equation}
%==========
%==========
\begin{equation} \label{250917180426} 
   B  = \frac{A}{\sqrt{2g}}
\end{equation}
%==========
を導入すると微分方程式
%==========
\begin{equation} \label{250917180438} 
   A = \left(1 -   \frac{B}{\sqrt{H_0-K_2(t)}}\right) \dot{K}_2(t)
\end{equation}
%==========
を得る。
これは変数分離型なので、
%==========
\begin{equation} \label{250917180856} 
   A  \mathrm{d}t =  \left(1 -   \frac{B}{\sqrt{H_0-K_2(t)}}\right) \mathrm{d} K_2
\end{equation}
%==========
とすると
%==========
\begin{equation} \label{250917181005} 
   A t + C =  K_2(t) + 2B\sqrt{H_0-K_2(t)}
\end{equation}
%==========
とりあえずいまは$t=0$で$K_2=0$という初期条件でこの微分方程式を解くと(あとで初期条件の時刻を$t=\tau_{f0}$に平行移動する)、
積分定数
%==========
\begin{equation} \label{250917180719} 
  C = 2B\sqrt{H_0}
\end{equation}
%==========
を得る。
したがって
%==========
\begin{equation} \label{250917181329} 
   A t + 2B\sqrt{H_0} = K_2(t) + 2B\sqrt{H_0-K_2(t)}
\end{equation}
%==========
$K_2$を変数変換し
%==========
\begin{equation} \label{250917181440} 
   S(t) =  \sqrt{H_0 - K_2(t)}
\end{equation}
%==========
つまり
%==========
\begin{equation} \label{250917181759} 
   K_2(t) = H_0 - S(t)^2
\end{equation}
%==========
とすると、
%==========
\begin{equation} \label{250917181553} 
   A t + 2B\sqrt{H_0}  = H_0 - S(t)^2 + 2BS(t)
\end{equation}
%==========
%==========
\begin{equation} \label{250917181633} 
   S(t)^2 - 2BS(t) + 2B\sqrt{H_0} + At - H_0
\end{equation}
%==========
これを解くと
%==========
\begin{equation} \label{250917181902} 
   S(t) = B \pm \sqrt{B^2-2B\sqrt{H_0} -At + H_0}
\end{equation}
%==========
%==========
\begin{equation} \label{250917182147} 
   S(t) = B\pm \sqrt{(\sqrt{H_0}-B)^2-At}
\end{equation}
%==========
$\sqrt{H_0}-B$の符号をみるために$B/\sqrt{H_0}$を評価すると
%==========
\begin{equation} \label{250917182519} 
   \frac{B}{\sqrt{H_0}} = \frac{K_0m\mu}{M_0\sqrt{2gH_0}} = \frac{K_0/T_0}{\sqrt{2gH_0}}
\end{equation}
%==========
右辺の分子は砂時計の速度スケール(砂層の初期高さと計時時間の比)で、分母は砂粒の(最大)自由落下速度である。したがって
%==========
\begin{equation} \label{250917183422} 
    \frac{B}{\sqrt{H_0}} \ll 1
\end{equation}
%==========
としてよい。つまり
%==========
\begin{equation} \label{250917183450} 
   \sqrt{H_0}-B >0
\end{equation}
%========== 
である。
$t=0$で$s=\sqrt{H_0}$になるので式\eqref{250917182147}の複号はプラスをとる
%==========
\begin{equation} \label{250917182410} 
   S(t) = B + \sqrt{(\sqrt{H_0}-B)^2-At}
\end{equation}
%==========
式\eqref{250917181759}より
%==========
\begin{equation} \label{250917183653} 
   K_2(t) = H_0 - \left\{
   				B + \sqrt{(\sqrt{H_0}-B)^2-At}
   				\right\}^2
\end{equation}
%==========


この解は複雑なので、式\eqref{250917183422}を用いて、近似を導入する。
%==========
\begin{equation} \label{250918054610} 
   \epsilon =  \frac{B}{\sqrt{H_0}}
\end{equation}
%==========
とする。
%==========
\begin{align}
   K_2(t) &= H_0 - \left\{
   				\epsilon \sqrt{H_0} + \sqrt{(\sqrt{H_0}-\epsilon \sqrt{H_0})^2-At}
   				\right\}^2   \label{250918054717a} \\
   &= H_0 - \left\{
   				\epsilon \sqrt{H_0} + \sqrt{H_0(1-\epsilon )^2-At}
   				\right\}^2   \label{250918054717b} \\
   &=   H_0 - H_0\left\{
   				\epsilon  + \sqrt{(1-\epsilon )^2-\frac{At}{H_0}}
   				\right\}^2  \label{250918054717c} \\
   &=    H_0 - H_0\left\{
   				 \sqrt{1-\frac{At}{H_0}}
   				\right\}^2  + \mathcal{O}(\epsilon)  \label{250918054717d}\\
   &=    H_0 - H_0\left(
   				 1-\frac{At}{H_0}
   				\right) + \mathcal{O}(\epsilon)  \label{250918054717e}\\
   &=    \frac{At}{H_0}
   				+ \mathcal{O}(\epsilon)  \label{250918054717f}
\end{align}
%==========
つまり$\mathcal{O}(\epsilon) $の誤差で微分方程式\eqref{250917180438}の解は
%==========
\begin{equation} \label{250918075020} 
   K_2(t) = \frac{At}{H_0}
\end{equation}
%==========
である。



この解は微分方程式\eqref{250917180438}を$t=0$で$K_2=0$という初期条件で解いたものであったが、本来は
$t=\tau_{f0}$で$K_2=0$とすべきであった。
つまり解は
%==========
\begin{equation} \label{250917184208} 
   K_2(t) =  \frac{A(t-\tau_{f0})}{H_0} = \frac{m \mu K_0}{M_0}(t-\tau_{f0})
\end{equation}
%==========
である。



\color{black}

下の砂層の重心の$y$座標は
%==========
\begin{equation} \label{250703154424} 
   Y_2(t) = -H_0 + \frac{K_2(t)}{2} 
\end{equation}
%==========



砂粒がオリフィスから物体2に到達するまでの時間(自由落下時間)を$\tau_f(t)$とし、
上の砂層を質量$M_1(t)$をもつ物体1、下の砂層を質量$M_2(t)$をもつ物体2、落下中の砂粒全体を物体3とみなす。


通常の物体と異なり、いずれの物体も質量は時間的に一定ではなく、
物体1の質量は式\eqref{250703152655}から常に減少しており、
物体2の質量は$t>\tau_f(t)$では常に増加する。
物体3の質量は$t<\tau_f(t)$では増加し、それ以降は減少する。


物体1の運動量は
%==========
\begin{equation} \label{250917102959} 
   P_1(t)  = M_1(t) \dot{Y}_1 
\end{equation}
%==========
物体2の質量は
%==========
\begin{equation} \label{250917184641} 
   M_2 = \frac{K_2(t)}{K_0} M_0
\end{equation}
%==========
物体2の運動量は
%==========
\begin{equation} \label{250917103019} 
   P_2(t) = M_2(t)\dot{Y}_2  
\end{equation}
%==========
である。


物体3の運動量分布は高さ$y$によらず一定であることに注意する。
位置$y$を速度$v_y(y)$で下方$(v_y<0)$に通過する砂粒の数は常に$\mu$個なので、砂粒の数密度を$n(y)$とすると、
%==========
\begin{equation} \label{250917101402} 
   n(y) v_y(t) = -\mu = \text{const.}
\end{equation}
%==========
である。したがって物体3の運動量の密度は
%==========
\begin{equation} \label{250917102631} 
   p_3 = mn(y)v_y(y)= -m \mu = \text{const.}
\end{equation}
%==========
は一定である。
したがって物体3の運動量は
%==========
\begin{equation} \label{250917102719} 
   P_3(t) = p_3 \times (H_0 - K_2(t)) = -m\mu(H_0-K_2(t)
\end{equation}
%==========



%
%
%オリフィスから落下した砂粒が時刻$t$において物体2の上面に衝突することで作用する力を計算する。
%%%==========
%%\begin{equation} \label{250703163145} 
%%   \tau_f(t) = \sqrt{\frac{2(H_0-K_2(t))}{g}}
%%\end{equation}
%%%==========
%落下時の速度$v_f(t)$は
%%==========
%\begin{equation} \label{250703161618} 
%   v_f(t) = g\tau_f(t)
%\end{equation}
%%==========
%である。
%物体2は1秒間に$\mu$回、この力積(下向きの力)を受ける。
%その時間平均を$F_p>0$とすると
%%==========
%\begin{equation} \label{250703161334} 
%  F_p(t)  = m v_f(t) \mu = m g \mu \tau_f(t)
%\end{equation}
%%==========
%である。
%$\mu\tau_f(t)$はこの時刻に落下途中の砂の数なので、
%%==========
%\begin{equation} \label{250703172127} 
%   F_p(t) = \left\{M_0-M_1(t) -M_2(t)\right\} g
%\end{equation}
%%==========
%である。
%


%物体1と物体2の重心の速度を$V_1(t)$と$V_2(t)$とすると
%%==========
%\begin{align}
%   V_1(t) &= \frac{\mathrm{d} Y_1(t)}{\mathrm{d} t} = - \frac{m \mu }{2\rho S_0}  \label{250703154817a} \\
%   V_2(t) &= \frac{\mathrm{d} Y_2(t)}{\mathrm{d} t} = + \frac{m \mu }{2\rho S_0}   \label{250703154817b} 
%\end{align}
%%==========
%である。



$F_1 (>0)$をオリフィスのある面が物体1を支える抗力、
$F_2 (>0)$を砂時計の下の床面が物体2を支える抗力、
物体1と物体3の間に作用する力を$F_{13}$、物体2と物体3の間に作用する力を$F_{32}$とすると、
物体1と物体2と物体3の運動方程式は
%==========
\begin{align}
   \frac{\mathrm{d} }{\mathrm{d} t}  P_1(t) &=   - M_1(t) g - F_{13} + F_1\label{250703155015a}  \\
   \frac{\mathrm{d} }{\mathrm{d} t} P_3(t) &=  - M_3(t) g  + F_{13} + F_{32} \label{2507031550153c} \\
   \frac{\mathrm{d} }{\mathrm{d} t} P_2(t)  &=  - M_2(t) g  - F_{32} + F_2 \label{250703155015b} 
\end{align}
%==========
砂時計全体を支える抗力$F$は$F=F_1+F_2$であることに注意する。


式\eqref{250703155015a}から\eqref{250703155015b}の両辺を足すと
%==========
\begin{equation} \label{250917103450} 
  \dot{M}_1 \frac{\dot{K}_1}{2}  + m \mu \dot{K}_2 + \dot{M}_2 \frac{\dot{K}_2}{2} =  -(M_1+M_2+M_3) g + F_1 + F_2
\end{equation}
%==========


したがって砂時計全体を支える抗力は
%==========
\begin{align}
   F &= F_1 + F_2  \label{250703170301a} \\
   &=(M_1(t) + M_2(t)  +M_3(t) )g +  2\frac{m^2\mu^2}{\rho S_0}   \label{250703170301b} \\
   &=  M_0 g +  2 \frac{m^2\mu^2}{\rho S_0}   \label{250703170301c} 
\end{align}
%==========
つまり計時中の砂時計は右辺第2項
%==========
\begin{equation} \label{250703173105} 
   \Delta W  = 2 \frac{m^2\mu^2}{\rho S_0} 
\end{equation}
%==========
だけ重くなる。
式\eqref{250515082817}と式\eqref{250703152145}より
%==========
\begin{equation} \label{250703173052} 
    \Delta W  =  2M_0 \frac{K_0}{T_0^2} 
\end{equation}
%==========
とも書ける。


%=============================================
\section{まとめ}
%=============================================
砂が落ちていないときの砂の層の厚さを$K_0$、砂の全質量を$M_0$、砂時計が測る時間を$T_0$とすると、砂が落下しているときの砂時計は、
%==========
\begin{equation} \label{250518122927} 
   \Delta W = 2 M_0 \frac{K_0}{T_0^2} 
\end{equation}
%==========
だけ重くなる。
計時中の砂時計の重心は明らかに下に移動し、重心の速度は下向きであるが、
その加速度が上向きであることがこの重さの起源である。
砂時計の高さ、つまり砂の落下する距離や砂時計のガラス部分の断面積には依存しないことは興味深い。


たとえば、$K=5$~(\si{cm}), $M_0=100$~(\si{g}), $T_0=10^2$~(\si{s}) とすると
%==========
\begin{equation} \label{250517205142} 
   \Delta W = -\times 10^{-6}~\si{kg}
\end{equation}
%==========
であり、この重さの差はかなり小さい。


\bibliographystyle{amsplain} 
\bibliography{hourglass_weight}



\end{document}