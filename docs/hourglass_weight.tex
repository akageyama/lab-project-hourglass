\documentclass[]{article}
\usepackage{siunitx}
\usepackage[dvipdfm]{hyperref}
\usepackage[dvipdfmx]{graphicx}
\usepackage{bm}
\usepackage{fancyhdr}
\usepackage{indentfirst}
\usepackage{listings}
\usepackage{amsmath,amssymb}
\usepackage{here}
\usepackage{ascmac}
\usepackage[dvipsnames]{xcolor}
\usepackage{url}
\usepackage{colortbl}
\usepackage{comment}

\title{砂時計の重さ\footnote{%
これは学部の1年生向けの講義で力積について教えた際に例題の一つとして挙げようとした話題である。
力積の計算から「重さは変わらない」という結論がでることがちょっと面白いであろうと思って準備をしていたが、
重心の移動を考えるとそれはおかしいということに講義の直前になって気がついた。
面白い問題であることは間違いない。
}}
\author{陰山 聡$^\dagger$, 中島 涼輔$^\dagger$, 中戸 昂明$^\ddagger$\\[0.5em] $^\dagger$神戸大学システム情報学研究科, $^\ddagger$神戸大学システム情報学部}
\date{\today}

\begin{comment}
\end{comment}

\begin{document}

\maketitle


\begin{abstract}
砂時計の砂が落ちている間、砂時計の重さは変わるであろうか?
落ちている最中の砂粒の重みは砂時計には作用しないので、その分だけ軽くなる一方、
下に落ちた砂は床面に力積を与えるのでその分だけ重くなる。
計算するとこの両者はちょうどキャンセルするので、重さは変わらないという結論になりそうであるが、
もう少し正確に計算すると、砂時計全体はわずかに重くなる。
\end{abstract}


%=============================================
\section{問題設定}
%=============================================
$T_0$秒間を測る砂時計を考える。3分計であれば$T_0=180$~(\si{s})である。
重力加速度を$g~(\si{m.s^{-2}})$、砂粒の総数を$N$、砂全体の質量を$M_0~(\si{kg})$とする。


砂粒一つの質量$m$は、
%==========
\begin{equation} \label{250517100232} 
   m = \frac{M_0}{N} 
\end{equation}
%==========
1秒間に落ちる砂粒の数を$\mu~(\si{s^{-1}})$とすると、
%==========
\begin{equation} \label{250515082817} 
   \mu = \frac{N}{T_0}
\end{equation}
%==========
である。
なお、下に落ちた砂粒は跳ね返らず、短い時間で静止すると仮定する。


%=============================================
\section{力積の評価による概算} \label{250517204115}
%=============================================

一つの砂粒が砂時計のくびれ部分(オリフィス)から落下し、下部に積み重なった砂の層の表面まで到達するのにかかる時間、つまり自由落下時間を$\tau_f$~(\si{s})とすると、
落下した瞬間の砂粒の速さ(自由落下速度)は
%==========
\begin{equation} \label{250512190520} 
   V_F = g\tau_f
\end{equation}
%==========
なので、落下によって静止した砂粒の運動量の変化、つまり力積は
%==========
\begin{equation} \label{250512190655} 
   p = mg\tau_f
\end{equation}
%==========
である。


$\Delta t$秒間に砂粒は$\mu \Delta t$個だけ落下する。
この$\Delta t$秒間に砂粒が下面を押す力の平均を$F_\mathrm{i}$~(\si{N})とすると
%==========
\begin{equation} \label{250512191540} 
  \Delta t \times F_\mathrm{i} = \mu \Delta t \times p = \mu \Delta t \times mg\tau_f
\end{equation}
%==========
つまり
%==========
\begin{equation} \label{250515093535} 
  F_\mathrm{i}   = \mu m g \tau_f
\end{equation}
%==========
である。


一方、ある瞬間に落下している途中、つまり砂時計のオリフィスから下面までの中空にいる砂粒は$\mu \tau_f$個ある。
一つの砂粒にかかる重力は$mg$なので、
砂時計全体の重さはこの落下中の砂粒の重さ
%==========
\begin{equation} \label{250513083619} 
   F_\mathrm{s} = \mu \tau_f\times mg 
\end{equation}
%==========
だけ軽くなる。
これは力積による平均力、つまり式~\eqref{250515093535}と等しい。
従って砂が落ちている最中の砂時計全体の重さは変わらない、という結論になりそうだが、厳密にはこれは正しくない。


砂時計全体の重心の運動について考えよう。
砂が落ち始める前と落ちきった後とでは明らかに砂時計全体の重心位置は下に移動している。
したがって重心には、はじめに下向きの力が作用し、その後、上向きの力でその下方向の速度を止めたはずである。
ただし、その重心移動の距離はわずかで、しかもかなり時間をかけたゆっくりとしたもの(砂時計が3分計であれば3分間)なので、
その加速度は必然的に小さく、したがって砂時計全体の重さに与える影響は小さいことは予想できる。


%=============================================
\section{運動方程式に基づいた計算}
%=============================================

砂時計の重心が移動する効果を考慮するためにはこの問題を静力学の問題としてではなく、運動方程式に基づいて考えるべきである。


鉛直上向きに$y$軸をとり、
オリフィスの位置を$y=0$とする。
オリフィスから砂時計のガラスの底面までの距離、
つまりガラス内部の空間の高さの半分を$H_0$~(\si{m})とする。
砂が移動するガラス内部の天井面と床面の位置は
それぞれ$y=H_0$と$y=-H_0$である。


オリフィス付近を除いた砂時計の断面積を$S_0$~(\si{m^2})、
計時を開始する前の砂の層の厚さを$K_0$~(m)、
砂粒の質量密度を$\rho$~(\si{kg/m^3})とすると
砂の全質量が$M_0$~(\si{kg})なので
%==========
\begin{equation} \label{250703152145} 
   M_0 = N m = S_0 K_0 \rho
\end{equation}
%==========
である。


計時開始時刻を$t=0$とすると、
時刻$t$では$\mu t$個の砂粒がオリフィスから落下しているので、
この時刻にオリフィスの上の部分にある砂、つまりまだ落下していない砂の数は
%==========
\begin{equation} \label{250703152453} 
   N_1(t) =  N - \mu t
\end{equation}
%==========
である。
したがってオリフィス上部の砂の質量は
%==========
\begin{equation} \label{250703152655} 
   M_1(t) = m (N-\mu t)
\end{equation}
%==========
である。
この時刻におけるオリフィス上部の砂層の厚さは
%==========
\begin{equation} \label{250703152748} 
   K_1(t) = \frac{M_1(t)}{\rho S_0} = \frac{m}{\rho S_0}(N-\mu t)
\end{equation}
%==========
である。
この層の重心の$y$座標を$Y_1(t)$とすると
%==========
\begin{equation} \label{250703153013} 
   Y_1(t) = \frac{K_1(t)}{2} = \frac{m}{2\rho S_0} ( N - \mu t)
\end{equation}
%==========


時刻$t=0$に落下した最初の砂粒が下の床に到達するまでの時間、つまり自由落下時間を$\tau_{f0}$とする
%==========
\begin{equation} \label{250703153227} 
   \tau_{f0} = \sqrt{\frac{2H_0}{g}}
\end{equation}
%==========


\color{blue}

%==========
\begin{equation} \label{250917175554} 
   M_1(t) = \frac{M_0}{K_0} K_1(t)
\end{equation}
%==========
%==========
\begin{equation} \label{250917175633} 
   M_2(t) =  \frac{M_0}{K_0} K_2(t)
\end{equation}
%==========

%==========
\begin{equation} \label{250917175724} 
   N_3(t) = \mu \tau_f(t) = \mu \sqrt{\frac{2(H_0-K_2(t))}{g}} 
\end{equation}
%==========
より
%==========
\begin{equation} \label{250917175839} 
   M_3(t) = m \mu \sqrt{\frac{2}{g}} \sqrt{H_0-K_2(t)}
\end{equation}
%==========
質量保存則より
%==========
\begin{equation} \label{250917175922} 
   M_0 = M_1(t) + M_2(t) + M_3(t)
\end{equation}
%==========
時間微分をとると、
%==========
\begin{equation} \label{250917180035} 
   -M_0 \dot{K}_1(t) = M_0 \dot{K}_2(t)  - \frac{m\mu K_0}{\sqrt{2g}} \frac{\dot{K}_2(t)}{\sqrt{H_0-K_2(t)}}
\end{equation}
%==========
%==========
\begin{equation} \label{250917180138} 
   \dot{K}_1 = -\frac{m\mu}{S_0\rho}
\end{equation}
%==========
と
%==========
\begin{equation} \label{250917180220} 
   M_0 =  S_0 K_0 \rho
\end{equation}
%==========
を代入すると
%==========
\begin{equation} \label{250917180242} 
   \frac{K_0 m \mu}{M_0} =  \left(1 - \frac{1}{\sqrt{2g}}\frac{m \mu K_0}{M_0}  \frac{1}{\sqrt{H_0-K_2(t)}}\right) \dot{K}_2(t)
\end{equation}
%==========
定数
%==========
\begin{equation} \label{250917180407} 
   A = \frac{m \mu K_0}{M_0}
\end{equation}
%==========
%==========
\begin{equation} \label{250917180426} 
   B  = \frac{A}{\sqrt{2g}}
\end{equation}
%==========
とすると微分方程式
%==========
\begin{equation} \label{250917180438} 
   A = \left(1 -   \frac{B}{\sqrt{H_0-K_2(t)}}\right) \dot{K}_2(t)
\end{equation}
%==========
を得る。
これは変数分離型なので解ける。
%==========
\begin{equation} \label{250917180856} 
   A  \mathrm{d}t =  \left(1 -   \frac{B}{\sqrt{H_0-K_2(t)}}\right) \mathrm{d} K_2
\end{equation}
%==========
とすると
%==========
\begin{equation} \label{250917181005} 
   A t + C =  K_2(t) + 2B\sqrt{H_0-K_2(t)}
\end{equation}
%==========
$t=0$で$K_2=0$という初期条件でこの微分方程式を解くと、(あとで初期条件の時刻を平行移動する)
%==========
\begin{equation} \label{250917180719} 
  C = 2B\sqrt{H_0}
\end{equation}
%==========
なので
%==========
\begin{equation} \label{250917181329} 
   A t + 2B\sqrt{H_0} = K_2(t) + 2B\sqrt{H_0-K_2(t)}
\end{equation}
%==========
変数変換して
%==========
\begin{equation} \label{250917181440} 
   s(t) =  \sqrt{H_0 - K_2(t)}
\end{equation}
%==========
つまり
%==========
\begin{equation} \label{250917181759} 
   K_2(t) = H_0 - s(t)^2
\end{equation}
%==========
とすると、
%==========
\begin{equation} \label{250917181553} 
   A t + 2B\sqrt{H_0}  = H_0 - s(t)^2 + 2Bs(t)
\end{equation}
%==========
%==========
\begin{equation} \label{250917181633} 
   s(t)^2 - 2Bs(t) + 2B\sqrt{H_0} + At - H_0
\end{equation}
%==========
これを解くと
%==========
\begin{equation} \label{250917181902} 
   s(t) = B \pm \sqrt{B^2-2B\sqrt{H_0} -At + H_0}
\end{equation}
%==========
%==========
\begin{equation} \label{250917182147} 
   s(t) = B\pm \sqrt{(\sqrt{H_0}-B)^2-At}
\end{equation}
%==========
$\sqrt{H_0}-B$の符号をみるために$\sqrt{H_0}/B$を評価すると
%==========
\begin{equation} \label{250917182519} 
   \frac{\sqrt{H_0}}{B} = \frac{M_0\sqrt{2gH_0}}{K_0m\mu} = \frac{\sqrt{2gH_0}}{K_0/T_0}
\end{equation}
%==========
右辺の分子は砂粒の(最大)自由落下速度で、分母は砂時計の速度スケール(砂層の初期高さと計時時間の比)である。したがって
%==========
\begin{equation} \label{250917183422} 
    \frac{\sqrt{H_0}}{B} \gg 1
\end{equation}
%==========
としてよい。つまり
%==========
\begin{equation} \label{250917183450} 
   \sqrt{H_0}-B >0
\end{equation}
%========== 
である。
$t=0$で$s=\sqrt{H_0}$になるので式\eqref{250917182147}の複号はプラスをとる
%==========
\begin{equation} \label{250917182410} 
   s(t) = B + \sqrt{(\sqrt{H_0}-B)^2-At}
\end{equation}
%==========
式\eqref{250917181759}より
%==========
\begin{equation} \label{250917183653} 
   K_2(t) = H_0 - \left\{
   				B + \sqrt{(\sqrt{H_0}-B)^2-At}
   				\right\}^2
\end{equation}
%==========
この解は$t=0$で$K_2=0$という初期条件で解いたものであったが、本来は
$t=\tau_{f0}$で$K_2=0$とすべきであった。
つまり解は
%==========
\begin{equation} \label{250917184208} 
   K_2(t) =  H_0 - \left\{
   				B + \sqrt{(\sqrt{H_0}-B)^2-A(t-\tau_{f0})}
   				\right\}^2  \quad (t\ge \tau_{f0})
\end{equation}
%==========
である。
これは複雑な関数だが、$t-\tau_{f0} \ll 1$のときには$t$の線形で増加する
%==========
\begin{align}
    K_2(t) &= H_0 - \left\{
    				   B + (\sqrt{H}_0-B)\sqrt{
				   					1-\frac{A(t-\tau_{f0})}{(\sqrt{H}_0-B)^2}
				   				}
    				\right\}^2  \label{250917190909a} \\
   &\approx H_0 - \left\{
    				   B + (\sqrt{H}_0-B)\left[
				   					1-\frac{A(t-\tau_{f0})}{2(\sqrt{H}_0-B)^2}
				   				\right]
    				\right\}^2    \label{250917190909b} \\
   &= H_0 - \left\{
    				   B + (\sqrt{H}_0-B)
				   					-\frac{A(t-\tau_{f0})}{\sqrt{H}_0-B}
    				\right\}^2    \label{250917190909c} \\
   &=  H_0 - H_0\left\{
    				   1
				   					-\frac{A(t-\tau_{f0})}{H_0-B\sqrt{H}_0}
    				\right\}^2   \label{250917190909d} \\
   &\approx
   	H_0 - H_0\left\{
    				   1
				   					- 2 \frac{A(t-\tau_{f0})}{H_0-B\sqrt{H}_0}
    				\right\}    \label{250917190909e} \\
   &=  2 \frac{A(t-\tau_{f0})}{1-B/\sqrt{H}_0} \quad \text{\textcolor{magenta}{この式変形要確認}}\label{250917190909f} 
\end{align}
%==========
あとで使うので$K_2$の時間微分を計算しておく
%%==========
%\begin{equation} \label{250917185513} 
%   \dot{K}_2(t) =   \left\{
%   				B + \sqrt{(\sqrt{H_0}-B)^2-A(t-\tau_{f0})}
%   				\right\}^2  
%				\frac{A}{\sqrt{(\sqrt{H_0}-B)^2-A(t-\tau_{f0})}}			
%\end{equation}
%%==========
%==========
\begin{equation} \label{250917190039} 
  \dot{K}_2(t)  = A   
				\frac{\left\{
   				B + \sqrt{(\sqrt{H_0}-B)^2-A(t-\tau_{f0})}
   				\right\}^2}{\sqrt{(\sqrt{H_0}-B)^2-A(t-\tau_{f0})}}	
\end{equation}
%==========

$A$と$B$を元に戻すと 
%==========
\begin{equation} \label{250917184334} 
   K_2(t) =  H_0 - \left\{
   				\frac{m \mu K_0}{M_0\sqrt{2g}} + \sqrt{\left(\sqrt{H_0}-\frac{m \mu K_0}{M_0\sqrt{2g}}\right)^2-\frac{m \mu K_0}{M_0}(t-\tau_{f0})}
   				\right\}^2  \quad (t\ge \tau_{f0})
\end{equation}
%==========

\color{black}

下の砂層の重心の$y$座標は
%==========
\begin{equation} \label{250703154424} 
   Y_2(t) = -H_0 + \frac{K_2(t)}{2} 
\end{equation}
%==========



砂粒がオリフィスから物体2に到達するまでの時間(自由落下時間)を$\tau_f(t)$とし、
上の砂層を質量$M_1(t)$をもつ物体1、下の砂層を質量$M_2(t)$をもつ物体2、落下中の砂粒全体を物体3とみなす。


通常の物体と異なり、いずれの物体も質量は時間的に一定ではなく、
物体1の質量は式\eqref{250703152655}から常に減少しており、
物体2の質量は$t>\tau_f(t)$では式\eqref{250703153950}にしたがって常に増加する。
物体3の質量は$t<\tau_f(t)$では増加し、それ以降は減少する。


物体1の運動量は
%==========
\begin{equation} \label{250917102959} 
   P_1(t)  = M_1(t) \dot{Y}_1 
\end{equation}
%==========
物体2の質量は
%==========
\begin{equation} \label{250917184641} 
   M_2 = \frac{K_2(t)}{K_0} M_0
\end{equation}
%==========
物体2の運動量は
%==========
\begin{equation} \label{250917103019} 
   P_2(t) = M_2(t)\dot{Y}_2  
\end{equation}
%==========
である。


物体3の運動量分布は高さ$y$によらず一定であることに注意する。
位置$y$を速度$v_y(y)$で下方$(v_y<0)$に通過する砂粒の数は常に$\mu$個なので、砂粒の数密度を$n(y)$とすると、
%==========
\begin{equation} \label{250917101402} 
   n(y) v_y(t) = -\mu = \text{const.}
\end{equation}
%==========
である。したがって物体3の運動量の密度は
%==========
\begin{equation} \label{250917102631} 
   p_3 = mn(y)v_y(y)= -m \mu = \text{const.}
\end{equation}
%==========
は一定である。
したがって物体3の運動量は
%==========
\begin{equation} \label{250917102719} 
   P_3(t) = p_3 \times (H_0 - K_2(t)) = -m\mu(H_0-K_2(t)
\end{equation}
%==========



%
%
%オリフィスから落下した砂粒が時刻$t$において物体2の上面に衝突することで作用する力を計算する。
%%%==========
%%\begin{equation} \label{250703163145} 
%%   \tau_f(t) = \sqrt{\frac{2(H_0-K_2(t))}{g}}
%%\end{equation}
%%%==========
%落下時の速度$v_f(t)$は
%%==========
%\begin{equation} \label{250703161618} 
%   v_f(t) = g\tau_f(t)
%\end{equation}
%%==========
%である。
%物体2は1秒間に$\mu$回、この力積(下向きの力)を受ける。
%その時間平均を$F_p>0$とすると
%%==========
%\begin{equation} \label{250703161334} 
%  F_p(t)  = m v_f(t) \mu = m g \mu \tau_f(t)
%\end{equation}
%%==========
%である。
%$\mu\tau_f(t)$はこの時刻に落下途中の砂の数なので、
%%==========
%\begin{equation} \label{250703172127} 
%   F_p(t) = \left\{M_0-M_1(t) -M_2(t)\right\} g
%\end{equation}
%%==========
%である。
%


%物体1と物体2の重心の速度を$V_1(t)$と$V_2(t)$とすると
%%==========
%\begin{align}
%   V_1(t) &= \frac{\mathrm{d} Y_1(t)}{\mathrm{d} t} = - \frac{m \mu }{2\rho S_0}  \label{250703154817a} \\
%   V_2(t) &= \frac{\mathrm{d} Y_2(t)}{\mathrm{d} t} = + \frac{m \mu }{2\rho S_0}   \label{250703154817b} 
%\end{align}
%%==========
%である。



$F_1 (>0)$をオリフィスのある面が物体1を支える抗力、
$F_2 (>0)$を砂時計の下の床面が物体2を支える抗力、
物体1と物体3の間に作用する力を$F_{13}$、物体2と物体3の間に作用する力を$F_{32}$とすると、
物体1と物体2と物体3の運動方程式は
%==========
\begin{align}
   \frac{\mathrm{d} }{\mathrm{d} t}  P_1(t) &=   - M_1(t) g - F_{13} + F_1\label{250703155015a}  \\
   \frac{\mathrm{d} }{\mathrm{d} t} P_3(t) &=  - M_3(t) g  + F_{13} + F_{32} \label{2507031550153c} \\
   \frac{\mathrm{d} }{\mathrm{d} t} P_2(t)  &=  - M_2(t) g  - F_{32} + F_2 \label{250703155015b} 
\end{align}
%==========
砂時計全体を支える抗力$F$は$F=F_1+F_2$であることに注意する。


式\eqref{250703155015a}から\eqref{250703155015b}の両辺を足すと
%==========
\begin{equation} \label{250917103450} 
  \dot{M}_1 \frac{\dot{K}_1}{2}  + m \mu \dot{K}_2 + \dot{M}_2 \frac{\dot{K}_2}{2} \textcolor{magenta}{+\fbox{???}} =  -(M_1+M_2+M_3) g + F_1 + F_2
\end{equation}
%==========


したがって砂時計全体を支える抗力は
%==========
\begin{align}
   F &= F_1 + F_2  \label{250703170301a} \\
   &=(M_1(t) + M_2(t)  +M_3(t) )g  \textcolor{magenta}{+\fbox{???}} +  2\frac{m^2\mu^2}{\rho S_0}   \label{250703170301b} \\
   &=  M_0 g +  2 \frac{m^2\mu^2}{\rho S_0} \textcolor{magenta}{+\fbox{???}}  \label{250703170301c} 
\end{align}
%==========
つまり計時中の砂時計は右辺第2項
%==========
\begin{equation} \label{250703173105} 
   \Delta W  = 2 \frac{m^2\mu^2}{\rho S_0} \textcolor{magenta}{+\fbox{???}}
\end{equation}
%==========
だけ重くなる。
式\eqref{250515082817}と式\eqref{250703152145}より
%==========
\begin{equation} \label{250703173052} 
    \Delta W  =  2M_0 \frac{K_0}{T_0^2} \textcolor{magenta}{+\fbox{???}}
\end{equation}
%==========
とも書ける。


%=============================================
\section{まとめ}
%=============================================
砂が落ちていないときの砂の層の厚さを$K_0$、砂の全質量を$M_0$、砂時計が測る時間を$T_0$とすると、砂が落下しているときの砂時計は、
%==========
\begin{equation} \label{250518122927} 
   \Delta W = 2 M_0 \frac{K_0}{T_0^2} \textcolor{magenta}{+\fbox{???}}
\end{equation}
%==========
だけ重くなる。
計時中の砂時計の重心は明らかに下に移動し、重心の速度は下向きであるが、
その加速度が上向きであることがこの重さの起源である。
砂時計の高さ、つまり砂の落下する距離や砂時計のガラス部分の断面積には依存しないことは興味深い。


たとえば、$K=5$~(\si{cm}), $M_0=100$~(\si{g}), $T_0=10^2$~(\si{s}) とすると
%==========
\begin{equation} \label{250517205142} 
   \Delta W = -\times 10^{-6}~\si{kg}
\end{equation}
%==========
であり、この重さの差はかなり小さい。

%
%%=============================================
%\section{メモ: 砂全体の重心を用いた計算との違いについて}
%%=============================================
%砂全体の重心の$y$座標を$Y(t)$、落下中の砂粒の数密度を$n(y)$ (\si{m^{-1}})とすると
%%==========
%\begin{equation} \label{250911130712}
%   Y(t) = \frac{1}{M_0}\left(M_1(t) Y_1(t) + \int_{-H_0+\tilde{K}_2(t)}^0m n(y) y\mathrm{d}y + \tilde{M}_2(t) \tilde{Y}_2(t)\right)
%\end{equation}
%%==========
%となる。ここで、
%%==========
%\begin{equation} \label{250911130817}
%  \tilde{M}_2(t) = M_2(t) + m\int_{-H_0}^{-H_0+\tilde{K}_2(t)}n(y)\mathrm{d}y = M_2(t) + m\mu(\tau_{f0}-\tau_f(t))
%\end{equation}
%%==========
%は、下の砂層の上面が上昇することで、より多くの落下中の砂粒が下の砂層につけ加わることにより、単位時間あたりの質量増分が $m\mu$ よりも大きくなることを勘定に入れた下の砂層の質量である。さらに、その効果を考慮した場合の下の砂層の厚さと重心の$y$座標は、
%%==========
%\begin{align}
%   \tilde{K}_2(t) &= \frac{\tilde{M}_2(t)}{\rho S_0} = K_2(t) + \frac{m\mu}{\rho S_0}(\tau_{f0}-\tau_f(t))  \label{250911174154a}\\
%   \tilde{Y}_2(t) &= -H_0 + \frac{\tilde{K}_2(t)}{2}= Y_2(t) + \frac{m\mu}{2\rho S_0}(\tau_{f0}-\tau_f(t))  \label{250911174154b}
%\end{align}
%%==========
%である。式\eqref{250513083619}を用いることで、$n(y)$ と $\tilde{K}_2(t)$ の陽な表式を求めることなく、式\eqref{250911130817}の中辺の積分を計算でき、その結果、式\eqref{250911130817}の最右辺が得られている。$\tilde{M}_2$ を用いると、砂全体の質量は
%%==========
%\begin{equation} \label{250911135650}
%   M_0 = M_1(t) + m\mu \tau_{f}(t) + \tilde{M}_2(t) = M_1(t) + m\int_{-H_0+\tilde{K}_2(t)}^0n(y) \mathrm{d}y + \tilde{M}_2(t)
%\end{equation}
%%==========
%を満たすことから、式\eqref{250911130712}は重心の定義として矛盾がないことを確認できる。
%
%   
%
%
%式\eqref{250911130712}を時間微分すると、式\eqref{250703154817a}、\eqref{250703154817b} より
%%==========
%\begin{align} \label{250911171454}
%   M_0\frac{\mathrm{d}Y(t)}{\mathrm{d}t} &= \frac{\mathrm{d}M_1(t)}{\mathrm{d}t} Y_1(t) + M_1(t)V_1(t) \notag\\
%   &\qquad- m n(-H_0+\tilde{K}_2(t)) (-H_0+\tilde{K}_2(t))\frac{\mathrm{d}\tilde{K}_2(t)}{\mathrm{d}t} \notag\\
%   &\qquad+ \frac{\mathrm{d}\tilde{M}_2(t)}{\mathrm{d}t} \tilde{Y}_2(t) + \tilde{M}_2(t)\tilde{V}_2(t)
%\end{align}
%%==========
%となる。ここで、
%%==========
%\begin{equation} \label{250911180311}
%   \tilde{V}_2(t) = \frac{\mathrm{d}\tilde{Y}_2(t)}{\mathrm{d}t} = V_2 - \frac{m\mu}{2\rho S_0}\frac{\mathrm{d}\tau_f(t)}{\mathrm{d}t}
%\end{equation}
%%==========
%である。式\eqref{250911171454} を整理するために、ある高さ $y$ における落下中の砂粒の (鉛直上向きを正とした) 速度 $u(y)$ を考える。定義より、$u(-H_0+\tilde{K}_2(t))=-v_f(t)$である。さらに、砂の落下は定常であり、$n(y)$ と $v(y)$ は時間変化しないことから、ある高さ $y$ における砂粒のフラックスが一定となる条件
%%==========
%\begin{equation} \label{250911152410}
%   n(y) (-u(y)) = \mu =\text{const.}
%\end{equation}
%%==========
%を得る。また、
%%==========
%\begin{equation} \label{250911151314}
%   -H_0+\tilde{K}_2(t) = -\frac{g}{2}\tau_f^2(t)
%\end{equation}
%%==========
%より、式\eqref{250703161618} を用いると
%%==========
%\begin{equation} \label{250911183411}
%   \frac{\mathrm{d}\tilde{K}_2(t)}{\mathrm{d}t} = -v_f(t)\frac{\mathrm{d}\tau_f(t)}{\mathrm{d}t}
%\end{equation}
%%==========
%を得る。式\eqref{250911152410}、\eqref{250911183411} を用いると、式\eqref{250911171454} は
%%==========
%\begin{align} \label{250911183814}
%   M_0\frac{\mathrm{d}Y(t)}{\mathrm{d}t} &= \frac{\mathrm{d}M_1(t)}{\mathrm{d}t} Y_1(t) + M_1(t)V_1(t) + m\mu (-H_0+\tilde{K}_2(t))\frac{\mathrm{d}\tau_f(t)}{\mathrm{d}t} \notag\\
%   &\qquad+ \frac{\mathrm{d}\tilde{M}_2(t)}{\mathrm{d}t} \tilde{Y}_2(t) + \tilde{M}_2(t)\tilde{V}_2(t)
%\end{align}
%%==========
%となる。さらに、式\eqref{250911130817}、\eqref{250911174154a}、\eqref{250911174154b}、\eqref{250911180311} を代入し、式\eqref{250703153950}、\eqref{250703154021}、\eqref{250703154424}、\eqref{250703154817b} を用いると、
%%==========
%\begin{align} \label{250911193832}
%   M_0\frac{\mathrm{d}Y(t)}{\mathrm{d}t} &= \frac{\mathrm{d}M_1(t)}{\mathrm{d}t} Y_1(t) + M_1(t)V_1(t) + \frac{\mathrm{d}M_2(t)}{\mathrm{d}t} Y_2(t) + M_2(t)V_2(t) \notag\\
%   &\qquad+\frac{m^2\mu^2}{\rho S_0}(\tau_{f0}-\tau_f(t))
%\end{align}
%%==========
%を得る。
%
%
%
%
%$M_1$, $M_2$, $Y_1$, $Y_2$ が $t$ の 1 次関数であることを考慮し、式\eqref{250911193832} をさらに時間で微分すると、
%%==========
%\begin{equation} \label{250911194752}
%   M_0\frac{\mathrm{d}^2Y(t)}{\mathrm{d}t^2} = 2\frac{\mathrm{d}M_1(t)}{\mathrm{d}t} V_1(t) + 2\frac{\mathrm{d}M_2(t)}{\mathrm{d}t} V_2(t) - \frac{m^2\mu^2}{\rho S_0}\frac{\mathrm{d}\tau_f(t)}{\mathrm{d}t}
%\end{equation}
%%==========
%となる。最終的に式\eqref{250703154817a}、\eqref{250703154817b}、\eqref{250703152655}、\eqref{250703153950}より
%%==========
%\begin{equation} \label{250911195249}
%   M_0\frac{\mathrm{d}^2Y(t)}{\mathrm{d}t^2} = \frac{m^2\mu^2}{\rho S_0}\left(2 - \frac{\mathrm{d}\tau_f(t)}{\mathrm{d}t}\right)
%\end{equation}
%%==========
%を得る。この式は、$\tau_f$ の時間微分の補正項を除いて、式\eqref{250703173105} と $2$ 倍だけ異なる。これは、落下中の砂粒の運動量の時間変化を考慮に入れていなかったことに起因すると考えられる。
%
%
%
%
%そこで、運動方程式\eqref{250703155015a}、\eqref{250703155015b}を以下のように変更する。
%%==========
%\begin{align}
%   \frac{\mathrm{d} }{\mathrm{d} t} \left[M_1(t) V_1(t)\right] &=   - M_1(t) g + F_1(t)  \label{250911195923a} \\
%   \frac{\mathrm{d}}{\mathrm{d}t}\int_{-H_0+\tilde{K}_2(t)}^0 m n(y) u(y) \mathrm{d} y &= - g\int_{-H_0+\tilde{K}_2(t)}^0 m n(y) \mathrm{d}y + F_p(t)  \label{250911195923b} \\
%   \frac{\mathrm{d} }{\mathrm{d} t} \left[\tilde{M}_2(t) \tilde{V}_2(t) \right]  &=  - \tilde{M}_2(t) g -F_p(t)+ F_2(t) \label{250911195923c} 
%\end{align}
%%==========
%ここで、式\eqref{250911195923b} は、落下中の砂粒全体の運動方程式である。これは、式\eqref{250911135650}と式\eqref{250911152410}を用いると
%%==========
%\begin{equation} \label{250911203626}
%   -\frac{\mathrm{d}}{\mathrm{d}t}\int_{-H_0+\tilde{K}_2(t)}^0 m \mu \mathrm{d} y = - \left[M_0-M_1(t)-\tilde{M}_2(t)\right] g + F_p(t)
%\end{equation}
%%==========
%となり、さらに式\eqref{250911174154a}と\eqref{250703154021} を用いると
%%==========
%\begin{equation} \label{250911204756}
%   \frac{m^2\mu^2}{\rho S_0}\left(1-\frac{\mathrm{d}\tau_f(t)}{\mathrm{d}t}\right) = - \left[M_0-M_1(t)-\tilde{M}_2(t)\right] g + F_p(t)
%\end{equation}
%%==========
%を得る。この式の両辺を $0$ として近似したものが式\eqref{250703172127} である。また、式\eqref{250911130817}、\eqref{250911180311}、\eqref{250703153950}、\eqref{250703154817b} を用いると、式\eqref{250911195923c} は
%%==========
%\begin{align} \label{250911213549}
%   \frac{\mathrm{d} }{\mathrm{d} t} \left[M_2(t) V_2(t) \right] &- \frac{m^2\mu^2}{2\rho S_0}\left[(t-\tau_f(t))\frac{\mathrm{d}^2\tau_f(t)}{\mathrm{d}t^2}+2\frac{\mathrm{d}\tau_f(t)}{\mathrm{d}t}-\left(\frac{\mathrm{d}\tau_f(t)}{\mathrm{d}t}\right)^2\right]\notag\\
%    &=  - \tilde{M}_2(t) g -F_p(t)+ F_2(t) 
%\end{align}
%%==========
%となる。式\eqref{250911195923a}、\eqref{250911204756}、\eqref{250911213549} の和をとり、\eqref{250703154817a}、\eqref{250703154817b}、\eqref{250703152655}、\eqref{250703153950}を用いると、
%%==========
%\begin{align} \label{250911230907}
%   \frac{m^2\mu^2}{\rho S_0}\left(2 - \frac{\mathrm{d}\tau_f(t)}{\mathrm{d}t}\right) &- \frac{m^2\mu^2}{2\rho S_0}\left[(t-\tau_f(t))\frac{\mathrm{d}^2\tau_f(t)}{\mathrm{d}t^2}+2\frac{\mathrm{d}\tau_f(t)}{\mathrm{d}t}-\left(\frac{\mathrm{d}\tau_f(t)}{\mathrm{d}t}\right)^2\right] \notag\\
%   & = -M_0 g + F(t) \notag\\
%   & =  M_0\frac{\mathrm{d}^2Y(t)}{\mathrm{d}t^2}
%\end{align}
%%==========
%を得る。式\eqref{250911195249} の補正項と若干の違いはあるが、運動方程式から同様の式を得ることができた。
%
%
%
%
%なお、式\eqref{250703161618}、\eqref{250911183411}、\eqref{250911174154a}、\eqref{250911174154a}を用いると
%%==========
%\begin{equation} \label{250911232221}
%   \frac{\mathrm{d}\tau_f(t)}{\mathrm{d}t} = \frac{m^2\mu^2/\rho S_0}{-mg\mu\tau_f(t)+m^2\mu^2/\rho S_0}
%\end{equation}
%%==========
%とわかる。

\end{document}