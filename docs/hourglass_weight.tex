 \documentclass[dvipdfmx]{article}
\usepackage{siunitx}
\usepackage[dvipdfmx]{graphicx}
\usepackage{bm}
\usepackage{indentfirst}
\usepackage{amsmath,amssymb}
\usepackage{color}

\renewcommand{\figurename}{図}
\renewcommand{\tablename}{表}
\renewcommand{\refname}{参考文献}
\renewcommand{\abstractname}{要旨}
\usepackage{snotez}
\setlength{\marginparwidth}{2cm} 
\setlength{\marginparsep}{1.0cm} 

% [uncomment-this-line-for-submit]\renewcommand{\baselinestretch}{1.5}  % 投稿時これを有効にする。

\title{砂時計の重さ\\[1em]
        Weight of an Hourglass}

\author{	陰山 聡\thanks{神戸大学システム情報学研究科},
		中島 涼輔\thanks{神戸大学システム情報学研究科},
		中戸 昂明\thanks{神戸大学システム情報学部 }\\[1em]
		Akira Kageyama\thanks{Graduate School of System Informatics, Kobe University, Japan}, 
		Ryosuke Nakashima\thanks{Graduate School of System Informatics, Kobe University, Japan}, and
		Komei Nakato\thanks{Department of System Informatics, Kobe University, Japan}		
		}
\date{2025年9月28日}

\begin{document}


%=========================================================================
\begin{abstract}
砂時計の重さは砂が落ちている間にわずかに重くなることが知られている。
この微小な変化をシミュレーションで定量的に再現した。
可変質量物体が結合した力学の系として砂時計を考えると、
砂の層の厚みが時間変化する効果と、
自由落下する砂粒の落下距離が時間の経過と共に短くなる二つの効果が重さの変化に同等に寄与することがわかる。
このことは、砂粒の水平方向の運動は重要ではなく、砂粒同士の衝突も含めた砂粒の鉛直方向の運動が本質的であり、さらに下に降り積もった砂層の堆積過程を高精度に解くことが砂時計の重さの変化を再現するのに必須であることを意味する。
我々はソフトスフィアモデルに基づく個別要素法シミュレーションモデルを開発した。
これは、複数の砂の柱においてそれぞれ一定の割合で砂粒が落下し、下部に堆積していくというモデルである。
Processing言語で実装したこのシミュレーションはノートPCで実行することが可能である。


\vspace{1em}
%The weight of an hourglass is known to increase slightly while the sand is falling. 
%We quantitatively reproduced this minute change in weight through simulation. 
%By describing the motion of an hourglass as a system of interacting variable-mass objects, 
%we demonstrated that two effects contribute equally to the change in weight: 
%the effect of the time-dependent change in the thickness of the sand layer, 
%and the effect of the decreasing fall distance of falling sand grains over time. 
%We then developed a discrete element method simulation model based on the soft sphere model. 
%In this model we assumed sand grains falling constantly from multiple pillars. 
%This simulation, implemented in the Processing language, can be run on a notebook PC.

The weight of an hourglass is known to increase slightly when the sand is falling. 
We quantitatively reproduced this minute change through simulation. 
Considering the hourglass as a mechanical system of coupled variable-mass objects,  
we find that two effects equally contribute to the weight change: 
the effect of the sand layers' thickness varying over time, 
and the effect of the falling distance of freely falling sand grains becomes shorter as time passes. 
Based on this analysis, we developed a discrete element method simulation model using the soft-sphere model. 
In this model, we assume sand grains fall constantly and accumulate in multiple pillars. 
This simulation, implemented in Processing, can be run on a notebook PC.

\end{abstract}
%=========================================================================

\vspace{2em}

\paragraph{キーワード}
砂時計、力学、個別要素法、可変質量系\\[1em]
Hourglass, Mechanics, Discrete Element Method, Variable mass system


\maketitle



%=============================================
\section{はじめに}
%=============================================
砂時計を手に持ってその重さを測る状況を想像する。
砂時計の砂が落ち始める前の砂時計全体の重さと、
砂がすべて落ちきった後の重さはもちろん変わらない。
では、砂が落ちている最中の砂時計の重さはどうであろうか。
この問題は古くから知られており、
砂時計は少しだけ重くなることがわかっている。
だが、それは手で感じるに弱すぎるほどわずかな差である。


Scientific American誌の記事においてGardnerが砂時計を使ったおもちゃを紹介したことがきっかけとなり、
砂時計の重さの問題が注目を集めたようである\cite{Gardner1966-eu}。
その後、落下中の砂粒にかかる重みと落下時に失う運動量がちょうどキャンセルするので、砂時計の重さは変わらないはずであるという指摘が(この問題は力学教育のよい題材になるであろうというコメントとともに)された\cite{Reid1967-jq}。
その議論は以下のように簡単なものである:
砂時計のオリフィスから質量$m$の砂粒が落下し、$\tau_f$だけの時間をかけて底面に到達するあいだ、その重力$mg$の分だけ砂時計は軽くなる。
1秒間に$\mu$個の砂が落ちる砂時計であれば$\mu \tau_f$個の砂が空中にあるので、砂時計は$\mu \tau_f mg$だけ軽くなる。
一方、一つの砂粒が底面に到達したときにもつ運動量は$mg\tau_f$なので、落下する砂粒達が1秒あたりに底面に与える力積、つまり力は$m g\tau_f\mu$であり、空中にある砂にかかる重力とちょうどキャンセルする。


落下する砂粒の力積と重力がキャンセルするので、砂時計の重さは変わらないというこの説明は直感的で説得力があるが、正確ではない。


砂の重心に注目すると、砂が落ちている間、砂時計の重さが変化しないはずはないことがわかる。
最初の砂が落ち始める前、オリフィスよりも上で静止状態にあった砂全体の重心は、
砂が落ち始めると下部に移動しはじめ、全ての砂が落ちきった後には砂時計の下部で静止状態になる。
下に「落ちて」きた重心を砂時計のガラス部分(あるいは砂時計を乗せた手)が受け止めて静止させたのだからその間は上向きの力、つまり重みが増えたはずである。


とはいえ、重心移動の距離はわずか(砂時計の高さ程度)であり、
しかもかなりゆっくりとした(その砂時計が3分計であれば3分間の)移動なので、
重心の加速度は必然的に小さく、したがって砂時計全体の重さに与える影響は小さいことが予想できる。
この弱い効果を始めて実験的に検証したのはShen等である\cite{ShenUnknown-pk}。
彼らはガラスの粉末を使った「砂」時計の重量を測定し、ガラス粉末の総質量の$10^{-3}$倍程度重くなることを示した。
% - % - % - % - % - % - % - % - % - % - % - % - % - % - % - % - % - % - % - % - % - % - % - % - 
%  メモ:彼らの論文には総質量は明記されていない。
%        時間が10s から15sで、砂のfluxが3.7 g/s であるという記述があるので、
%        総質量は 35g から55.5 g と推測される。
%        重さの変化は、理論予想が49~\si{mg}に対して
%        実験結果は$30\sim 80$~\si{mg}の範囲と書いてあるので10^{-3}とした。
% - % - % - % - % - % - % - % - % - % - % - % - % - % - % - % - % - % - % - % - % - % - % - % - 


砂の代わりに金属の粉末を使った実験で、Sack等は砂時計の重さの時間変化を詳細に解析した\cite{Sack2017-rq}。
時間変化は主に3つの時期に分けることができる。
はじめに、砂が落ち始めた直後から最初の砂粒が下面に到達するまでの短い間、砂時計の重量は軽くなる。
これは空中にいる砂粒の重力分だけ重みがかからないためである。
逆に最終期、最後の砂粒が落ちている短い間、砂時計は重くなる。
これは下面を叩く砂粒達が与える力積をキャンセルするだけの数の砂が空中にないためである。
この二つの短い遷移時間の間、砂時計の重量がわずかに重くなる時期が長く続く。


砂時計の重さの問題は、理論的には単純そうでありながら、実際にはそうではないという点で興味深く、
また、質量が時間変化する力学の系の問題として教育的にも重要であると指摘されている\cite{Kassandrov2023-tn}。


期待される重量の変化がわずかなので、実験検証にさまざまな工夫が必要となる点でもこの問題は興味深い。
その中で特に注目されるのは、Tuinstra等の実験である\cite{Tuinstra2010-wk}。
彼らは通常の砂時計がもつようなくびれをもたない円筒状の砂時計で実験を行った。
単一の穴(オリフィス)の代わりに、
多数(230個)の穴をもつ「ふるい」のような板を設置し、多数の砂を同時に落下させた。
そして理論的に期待された通りの重さの増加を測定した。
%(本物の)砂を使った実験を行い、理論から予想される通りの重さの増加を測定した
% - % - % - % - % - % - % - % - % - % - % - % - % - % - % - % - % - % - % - % - % - % - % - % - % - % - % - % - % - % - % - % - 
% 論文によれば、M0=1.6$~\si{kg} である
% 理論的には$0.116$~\si{g}の重さの増加が期待され、測定結果は$0.121$~\si{g}であった。
% - % - % - % - % - % - % - % - % - % - % - % - % - % - % - % - % - % - % - % - % - % - % - % - % - % - % - % - % - % - % - % - 



コンピュータシミュレーションでこの効果、つまり砂が落下している最中の砂時計がわずかに重くなる効果を実証した研究はまだないようである。
本研究の目的は、落下中の砂の分だけ軽くなる効果と落下した砂が与える力積がちょうどキャンセルするという基本的な効果を再現することはもちろん、
砂の重心移動によるわずかな重さの変化という微弱な効果も定量的に再現するコンピュータシミュレーションを行うことである。
このシミュレーションプログラムはProcessing言語\footnote{\texttt{https://processing.org}}で書き、
ノートPCで実行した。



%=============================================
\section{理論}\label{250919151350}
%=============================================

%------------------------------------------
\subsection{問題設定} 
%------------------------------------------

$T_0$秒間を測る砂時計を考える。
重力加速度を$g$、砂粒の総数を$N_0$、砂全体の質量を$M_0$とする。
砂粒一つの質量$m$は、
%==========
\begin{equation} \label{250517100232} 
   m = \frac{M_0}{N_0} 
\end{equation}
%==========
である。
1秒間に落ちる砂粒の数を
%==========
\begin{equation} \label{250515082817} 
   \mu = \frac{N_0}{T_0}
\end{equation}
%==========
とする。
$\mu$が一定となるのは砂時計の重要な性質であり、オリフィスの上部でアーチ構造をつくる粉体の特性に起因する。
%$m$と$\mu$は定数である。


% . . . . . . . . . . . . . . . . . . . . . . . . . . . . . . . . . . . . . . . . 
      \marginnote{図\ref{250919103231}}
% . . . . . . . . . . . . . . . . . . . . . . . . . . . . . . . . . . . . . . . . 
図\ref{250919103231}の左に示すように、鉛直上向きに$y$軸をとり、
オリフィスの位置を$y=0$、
砂時計の底面の位置を$y=-H_0$とする。
最初の砂粒が落下する距離は$H_0$である。



砂時計の床面に働く抗力を求めるために運動方程式を立てる。
図\ref{250919103231}の右に示すように、
オリフィスよりも上にある砂の層を質量$M_1(t)$をもつ物体1、
下に落下して静止した砂の層を質量$M_2(t)$をもつ物体2、
落下中の砂粒全体を質量$M_3(t)$をもつ物体3とみなす。
つまり可変質量物体が相互作用する力学の問題として砂時計を考える。



砂時計の断面積は$y$によらず一定と仮定し、
計時を開始する前の砂の層の厚さを$K_0$、
砂粒の高さあたりの質量密度を$\sigma$とすると
%==========
\begin{equation} \label{250703152145} 
   \sigma = \frac{M_0}{K_0} %= \frac{mN_0}{K_0} 
\end{equation}
%==========
である。また、
%==========
\begin{equation} \label{250919090329} 
   \frac{m\mu}{\sigma} = \frac{K_0}{T_0}
\end{equation}
%==========
という関係がある。

%------------------------------------------
\subsection{物体1(落下前の上の砂層)}
%------------------------------------------

計時開始時刻を$t=0$とすると、
時刻$t$では$\mu t$個の砂粒がオリフィスから落下しているので、
この時刻にオリフィスの上の部分にある砂、つまりまだ落下していない砂の数は
%==========
\begin{equation} \label{250703152453} 
   N_1(t) =  N_0 - \mu t
\end{equation}
%==========
である。
したがってオリフィス上部の砂の質量は
%==========
\begin{equation} \label{250703152655} 
   M_1(t) = m (N_0-\mu t)
\end{equation}
%==========
である。
この時刻におけるオリフィス上部の砂層の厚さは
%==========
\begin{equation} \label{250703152748} 
   K_1(t) = \frac{M_1(t)}{\sigma} = \frac{m}{\sigma} (N_0-\mu t)
\end{equation}
%==========
である。
この式からわかるように、上の層の厚さは一定の割合
%==========
\begin{equation} \label{250919084914} 
   \dot{K}_1(t) = -\frac{m\mu}{\sigma}
\end{equation}
%==========
で減少する。
この層の重心の$y$座標を$Y_1(t)$とすると
%==========
\begin{equation} \label{250703153013} 
   Y_1(t) = \frac{K_1(t)}{2} = \frac{m}{2\sigma} (N_0-\mu t)
\end{equation}
%==========
である。物体1の運動量は
%==========
\begin{equation} \label{250917102959} 
   P_1(t)  = M_1(t) \dot{Y}_1(t) 
\end{equation}
%==========
その時間微分は
%==========
\begin{equation} \label{250919104731} 
   \dot{P}_1(t) = \dot{M}_1(t)\dot{Y}_1(t) = \frac{m^2\mu^2}{2\sigma}
\end{equation}
%==========
である。
運動量$P_1$は下向き($P_1<0$)であるが、その時間変化$\dot{P}_1$は上向き($\dot{P}_1>0$)であることに注意する。


%------------------------------------------
\subsection{物体2(落下後の下の砂層)}
%------------------------------------------

同様に下の砂層の厚さを$K_2(t)$、質量を$M_2(t)$とする。
これらの量が正になるのは時刻$t=0$に落下した最初の砂粒が底面に到達するまでの時間、
つまり自由落下時間
%==========
\begin{equation} \label{250703153227} 
   \tau_0 = \sqrt{\frac{2H_0}{g}}
\end{equation}
%==========
以降である。
時刻$t>\tau_0$では下の層に砂粒が$\mu$の割合で降り注ぐので、
%==========
\begin{equation} \label{250919091147} 
   M_2(t) = m\mu (t-\tau_0)
\end{equation}
%==========
と書ける。
したがって下の砂層の厚さは
%==========
\begin{equation} \label{250919104425} 
   K_2(t) = \frac{M_2(t)}{\sigma} = \frac{m\mu}{\sigma}(t-\tau_0)
\end{equation}
%==========
その時間微分は
%==========
\begin{equation} \label{250920171606} 
  \dot{K}_2(t)   = \frac{m\mu}{\sigma}
\end{equation}
%==========
である。


Appendix~\ref{250919092352}の式~\eqref{250920170423}に示すように、
実は式\eqref{250920171606}には$\mathcal{O}(\epsilon)$だけ誤差がある。
%正しくは
%%==========
%\begin{equation} \label{250920171723}
%   \dot{K}_2  = \frac{m\mu}{\sigma}\left( 1+ \frac{\epsilon}{2\sqrt{1-\beta  (t-\tau_0)}}\right)+ \mathcal{O}(\epsilon^2)
%\end{equation}
%%==========
%である。
ここで
%==========
\begin{equation} \label{250919092251} 
   \epsilon = 2\frac{K_0/T_0}{\sqrt{2gH_0}}  \ll 1
\end{equation}
%==========
は砂層の厚さが変化する速度(遅い)と砂粒の自由落下速度(速い)の比(の2倍)なので無視できる。
%したがって式\eqref{250920171723}の右辺の$\epsilon$に比例する項は無視してよい。
%運動量の時間変化に与えるこの誤差の影響は小さい。
%%==========
%\begin{equation} \label{250920161509} 
%    \dot{P}_2 = \dot{M}_2\frac{\dot{K_2}}{2} = \frac{\sigma}{2}(\dot{K}_2)^2 
%    =  \frac{\sigma}{2} \frac{m^2\mu^2}{\sigma^2}\left(1+ \frac{\epsilon}{\sqrt{1-\beta t}}\right) + \mathcal{O}(\epsilon^2)
%\end{equation}
%%==========
%%==========
%\begin{equation} \label{250923165505}
%    \dot{P}_2
%    	 =  \frac{m^2\mu^2}{2\sigma}\left(1 + \frac{\epsilon}{\sqrt{1-\beta t}}\right) + \mathcal{O}(\epsilon^2)
%\end{equation}
%%==========



この近似の下で、物体2の重心の$y$座標は
%==========
\begin{equation} \label{250703154424} 
   Y_2(t) = -H_0 + \frac{K_2(t)}{2} 
\end{equation}
%==========
であり、運動量は
%==========
\begin{equation} \label{250917103019} 
   P_2(t) = M_2(t)\dot{Y}_2(t)  
\end{equation}
%==========
その時間微分は
%==========
\begin{equation} \label{250919104243} 
   \dot{P}_2(t) = \dot{M}_2(t)\dot{Y}_2(t) = \frac{m^2\mu^2}{2\sigma}
\end{equation}
%==========
である。



%------------------------------------------
\subsection{物体3(落下中の砂粒)}
%------------------------------------------
上に述べた近似の下で、
$t\ge \tau_0$のある時刻$t$に落下している最中の砂粒は、
高さ($y$の範囲)
%==========
\begin{equation} \label{250919083726} 
   K_3(t) = H_0 - K_2(t)
\end{equation}
%==========
に分布している。


この物体3の運動量を評価するためにまず、物体3内部の運動量分布が高さ$y$によらず一定であることに注意する。
位置$y$を速度$v_y(y)$で下方$(v_y<0)$に通過する砂粒の数は単位時間あたり$\mu$個なので、砂粒の数密度を$n(y)$とすると、
%==========
\begin{equation} \label{250917101402} 
   n(y) v_y(y) = -\mu = \text{const.}
\end{equation}
%==========
である。したがって物体3の運動量密度
%==========
\begin{equation} \label{250917102631} 
   p_3 = mn(y)v_y(y)= -m \mu = \text{const.}
\end{equation}
%==========
は一定である。
したがって物体3の運動量は
%==========
\begin{equation} \label{250917102719} 
   P_3(t) =p_3 K_3(t) =  -m\mu(H_0-K_2(t))
\end{equation}
%==========
であり、
その時間変化は
%==========
\begin{equation} \label{250919105348} 
   \dot{P}_3(t)= m\mu\dot{K}_2(t) = \frac{m^2\mu^2}{\sigma}
\end{equation}
%==========
である。
ここでもまた$P_3$は下向きであるが、$\dot{P}_3$は上向きであることに注意する。


%$\mathcal{O}(\epsilon)$の補正の影響は小さい。
%%==========
%\begin{equation} \label{250920161728} 
%   \dot{P}_3 = m\mu \dot{K}_2 = \frac{m^2\mu^2}{\sigma}\left( 1 + \frac{\epsilon}{2\sqrt{1-\beta t}}\right) + \mathcal{O}(\epsilon^2)
%\end{equation}
%%==========
%%したがって
%%%==========
%%\begin{equation} \label{250920162242} 
%%   \dot{P}_2 + \dot{P}_3 =  \frac{m^2\mu^2}{\sigma} \left( \frac{3}{2} + \frac{\epsilon}{\sqrt{1-\beta t}}\right) + \mathcal{O}(\epsilon^2)
%%\end{equation}
%%%==========


%------------------------------------------
\subsection{運動方程式}
%------------------------------------------


砂時計の構造体(ガラス)が$y=0$の高さで物体1(上の砂層)を支える抗力を$R_1 (>0)$とし、
また、砂時計の構造体が$y=-H_0$で物体2(下の砂層)を支える抗力を$R_2 (>0)$とする。
抗力の和$F=R_1+R_2$が、
砂全体にかかる重力$M_0 \, g$よりも大きければ砂時計は重くなる。


物体1と物体3の間の相互作用および、物体2と物体3の間の相互作用をそれぞれ力$F_{13}$と力$F_{23}$で表現する。
$F_{13}$は存在したとしても微弱であろう。
$F_{23}$については後述するが、いずれにせよこの二つの力は砂時計全体からみれば内力なので、その具体的な表式は不要である。


3つの物体からなるこの系の運動方程式は
%==========
\begin{align}
  \dot{P}_1(t) &=   - M_1(t) g + F_{13} + R_1\label{250703155015a}  \\
  \dot{P}_2(t)  &=  - M_2(t) g  - F_{23} + R_2 \label{250703155015b} \\
  \dot{P}_3(t) &=  - M_3(t) g  - F_{13} + F_{23} \label{250703155015c} 
\end{align}
%==========
と書ける。


上の3つの式の両辺を足し、
式\eqref{250919104731}, \eqref{250919104243}, \eqref{250919105348}を使うと
%==========
\begin{equation} \label{250917103450} 
  2\frac{m^2\mu^2}{\sigma}=  -(M_1+M_2+M_3) g + R_1 + R_2
\end{equation}
%==========
となる。したがって
%==========
\begin{equation} \label{250919105951} 
   F = R_1 + R_2 = M_0 g +  2\frac{m^2\mu^2}{\sigma}
\end{equation}
%==========
を得る。つまり、計時中の砂時計が砂を支える抗力は、右辺第2項
%==========
\begin{equation} \label{250703173105} 
   \Delta F  = 2 \frac{m^2\mu^2}{\sigma}  = 2 \frac{M_0K_0}{T_0^2} 
\end{equation}
%==========
だけ大きくなる。
最後の式変形では式\eqref{250919090329}と$m\mu=M_0/T_0$を使った。
力ではなく重さで表現すれば
%==========
\begin{equation} \label{250703173052} 
    \Delta W  =  2 \frac{M_0K_0}{gT_0^2} 
\end{equation}
%==========
である。


落下する砂粒が床に与える力積が空中にある砂粒の重量とキャンセルするので砂時計の重さは変わらない、という冒頭に述べた議論は、
運動方程式\eqref{250703155015a}--\eqref{250703155015c}において、
左辺と$F_{13}$を無視した上で、$F_{23}=m\mu g\tau_f$と評価したことに相当する。



%
%Tuinstra等はくびれのない円筒形の容器を作り、その真ん中に設置した多数の穴をもつ「ふるい」状の板を通過して落下する砂柱の同時落下で実験を行った\cite{Tuinstra2010-wk}。
%その実験では$M_0=1.6~\si{(kg)}$、$T_0=23.6~\si{(s)}$、$K_0=0.198~\si{(m)}$である。
%%
%% --------------------------------------------------------------------------------------------------------------------------------
%% メモ:彼らの論文では 
%%      M = 1.6 (kg)
%% の砂を使ったとかかれているがTとK (彼らはLと書いている)の値は明記されていない。その代わり、測定値から
%%     M/T = 67.8 \times 10^{-3} (kg/s)
%%     L/T = 0.84 \times 10^{-2} (m)
%%  と書かれている。L/Tは時間の経過にともなって砂の層が薄くなるその速さを測定している。
%%  同様にM/Tも実測している。巧妙である。この3つの値からM, L(=K_0), Tを計算したのが上の値である。
%% --------------------------------------------------------------------------------------------------------------------------------
%式\eqref{250703173052}から評価される重さの変化は$\Delta W = 1.16\times 10^{-4}~\si{(kg)}$である。
%彼らの測定値は$\Delta W = (1.21\pm 0.06)\times 10^{-4}~\si{(kg)}$であった。
%相対的にこれほど微小な重さの差を検出するためには十分高い精度をもつ重量計と洗練された実験技術が必要である。



%=============================================
\section{シミュレーション}
%=============================================
%------------------------------------------
\subsection{砂粒のモデル}
%------------------------------------------

前章で述べた理論は、砂粒の水平方向の運動は本質的ではないこと及び、
砂の堆積効果を含めた鉛直方向の運動が重要であることを意味する。
そこで、各粒子が鉛直方向にのみ運動する空間1次元の粒子法(個別要素法)に基づくシミュレーションモデルを構築する。
落下して下の砂層に到達した砂粒は水平方向に動くことはなく、落ちた順番に上に積み上がっていく。
各砂粒は上下両隣の砂粒とだけ接触相互作用を考えればよいのがこのモデルの利点である。


% . . . . . . . . . . . . . . . . . . . . . . . . . . . . . . . . . . . . . . . . 
      \marginnote{図\ref{251120173401}}
% . . . . . . . . . . . . . . . . . . . . . . . . . . . . . . . . . . . . . . . . 
砂粒同士の相互作用および砂粒と床面との相互作用は図\ref{251120173401}(a)に示したソフトスフィアモデルで表現する。
各砂粒は、半径$a$、質量$m$の球とし、
それぞれ自然長$a$、ばね定数$k$のバネをもつ。
隣り合う二つの砂粒の中心間距離が$2a$以下になった場合、二つの粒子が接触していると判定する(図\ref{251120173401}(b)参照)。
二つのバネが接触しているときには線形バネによる反発力が生じる。
また、おなじく二つのバネが接触しているときには二つの粒子の相対速度の大きさ$v$に比例して$bv$の抵抗力をもつダンパーがはたらく。
粒子間距離が$2a$よりも大きい場合は互いに力をおよぼすことはない。


砂時計では通常、オリフィスから落下した一粒の砂粒が下に堆積した他の砂粒に高速で衝突したときもほとんど跳ね返ることなく静止する。
完全非弾性衝突に近いこのような現象をソフトスフィアモデルで再現するためには適切な$k$と$b$を設定するだけでなく、
十分に細かい時間刻み幅$\Delta t$をとる必要がある。
本シミュレーションでは時間積分法として陽的4次ルンゲ・クッタ法を用いる。


最初の砂粒はオリフィスの位置から距離$H_0$を自由落下し、速度
%==========
\begin{equation} \label{250919160111} 
   v_0 = \sqrt{2gH_0}
\end{equation}
%==========
で砂時計の床面に衝突し、静止する。
本シミュレーションでは重力加速度$g$を$g=9.80665$~\si{(m/s)}とした。
砂粒の半径と自由落下速度できまる時間スケール
%==========
\begin{equation} \label{250919160801} 
   t_0 = \frac{a}{v_0}
\end{equation}
%==========
に基づいてシミュレーションの時間刻み幅を
%==========
\begin{equation} \label{250919160045} 
   \Delta t = \alpha\, t_0
\end{equation}
%==========
とした。
ここで係数$\alpha$は十分小さい値で、本シミュレーションでは$\alpha=10^{-3}$とした。
バネ定数$k$は単振動の周期が$t_0$と等しくなるよう
%==========
\begin{equation} \label{250919161147} 
   k = m\left(\frac{2\pi}{t_0}\right)^2
\end{equation}
%==========
と設定した。
ダンパーの減衰係数$b$は
%==========
\begin{equation} \label{250919162003} 
   b = \gamma\, b_0
\end{equation}
%==========
として与える。
ここで$b_0=2\sqrt{mk}$は臨界減衰係数で、
$\gamma$は定数である。
本シミュレーションでは$\gamma=10$とした。


% . . . . . . . . . . . . . . . . . . . . . . . . . . . . . . . . . . . . . . . . 
      \marginnote{図\ref{250921122102}}
% . . . . . . . . . . . . . . . . . . . . . . . . . . . . . . . . . . . . . . . . 
図\ref{250921122102}に半径$a=1\times 10^{-2}$~\si{(m)}の粒子を高さ$y=1$~\si{(m)}から自由落下させたテストの結果を示す。
通常の砂時計ではあり得ないこれほどの高さから落下した砂も完全非弾性衝突、衝突すた砂粒がすなわち跳ね返らずに静止する計算が可能であることを確認するためのテストである。


現実の砂時計をよく観察すると、計時開始直後に落下した砂粒は床面のガラスに直接あたると跳ね返るのが観察されるが、
やがてすぐに砂粒の堆積層ができ、その堆積層の上面に落ちてくる砂粒は跳ね返らない。
砂粒のモデルではこのような完全非弾性衝突が再現される必要がある。


図\ref{250921122102}の左のグラフが示すように、粒子が放物線を描いて落下し、$y=0$に設定した床面に衝突して跳ね返ることなく止まる。
右のグラフは衝突の瞬間近くの拡大図である。
粒子の中心位置が$y=a$(右のグラフの灰色水平線)に達したときに床面と接触する。
この計算の時間刻み幅は$\Delta t = 2.258\times 10^{-6}$~\si{(s)}である。


粒子が床に接触するとバネによる反発力とダンパーによる抵抗が落下運動を減速させ、最終的には粒子を静止させることがこのように確認できた。
その最終状態は粒子の自重力($mg$)とバネによる反発力(=その反作用としての床からの抗力$R$)がつりあった平衡状態である。
床からの抗力は床と接触する粒子の下側のバネの自然長からのずれ(縮み)で決まる。
図\ref{251120173401}(c)に示したように、砂粒が多数積み重なった場合も同じである。




%------------------------------------------
\subsection{砂時計のモデル}
%------------------------------------------

% . . . . . . . . . . . . . . . . . . . . . . . . . . . . . . . . . . . . . . . . 
      \marginnote{図\ref{250924103856}}
% . . . . . . . . . . . . . . . . . . . . . . . . . . . . . . . . . . . . . . . . 
式\eqref{250703173052}が示すように、
砂時計の重さの増分$\Delta W$は、砂層の厚さ$K_0$に比例し、計時時間$T_0$の2乗に反比例する。
つまり、この微弱な量をシミュレーションで測定するためには、できるだけ厚い砂層の砂をできるだけ短時間に落とす必要がある。
また、粒子数が有限個であるために必然的に生じる時間的な揺らぎを小さくするためには、シミュレーションで追跡する粒子の数$N$は大きい方が望ましい。


多数の砂粒を鉛直方向に積み重ねるだけの単純な1次元モデル(図\ref{250924103856}左を参照)には限界がある。
鉛直方向に積み重なった砂粒が一定の時間間隔で下から順番に一粒ずつ落下する状況を考えると、
次の砂粒がオリフィスのある面に到達するためには粒子の直径$2a$の距離だけ自由落下する時間$\tau_a=2\sqrt{a/g}$程度は必要である。
この時間は一連の砂粒が落下する時間間隔$T_0/N$よりも小さくなければならない。
つまり $2\sqrt{a/g} < T_0/N$である。
%つまり
%%==========
%\begin{equation} \label{250924102819} 
%   \tau_a =2\sqrt{\frac{a}{g}} < \frac{T_0}{N}
%\end{equation}
%%==========
%であるが、これは
%%==========
%\begin{equation} \label{250924103332} 
%   4\frac{a}{g} < \frac{T_0^2}{N^2}
%\end{equation}
%%==========
%%==========
%\begin{equation} \label{250924103406} 
%   2 \frac{K_0}{N}  < \frac{gT_0^2}{N^2}
%\end{equation}
%%==========
%%==========
%\begin{equation} \label{250924103451} 
%  2\frac{K_0}{T_0^2}  < \frac{g}{N}
%\end{equation}
%%==========
これは粒子の数には上限があること
%==========
\begin{equation} \label{250924134548}
  N < \frac{gT_0^2}{2K_0}
\end{equation}
%==========
を意味する。


%また、たとえ粒子数$N$が式\eqref{250924134548}の上限以内であったとしても、
%$N$に反比例して$a$が減少する単純な1次元モデルの場合、式\eqref{250919160045}にしたがって$\Delta t$が小さくなるので、シミュレーションの実効計算速度が遅くなる。



砂粒の半径$a$と砂の層の厚さ$K_0$を保ちつつ、可能な限り多数の砂粒が$T_0$の時間内に落下するシミュレーションを実現するために、
この研究では図\ref{250924103856}右に示すように複数($N_\mathrm{P}$本)の砂の柱(sand pile)を設定した。
1本の砂柱は1次元的に(鉛直方向のみに)運動する砂粒($N_\mathrm{S}$個)から構成され、
一定の時間間隔で各砂柱から一粒ずつ砂粒が落下する。
この工夫により、同じ砂層の高さ$K_0$と同じ計時時間$T_0$であっても、短い時間間隔に多数の粒子を落下させることができる。



砂時計は二つの床面をもつ「2階建て」構造をもつ。
床面は$y=0$と$y=-H_0$にあり、ここでは前者をオリフィス (Orifice) のある面であることから$\mathrm{O}$ 、後者を底面 (Bottom) $\mathrm{B}$と呼ぶ。
最初の砂粒が落ち始める時刻に、オリフィス面$\mathrm{O}$は全部で$N_\text{p}$本の砂柱を支えている。
各砂柱には固有の整数$p$ ($0\le p \le N_\text{p}-1$) を割り当てる。
一本の砂柱は$N_\text{s}$個の砂粒から構成され、
砂粒には固有の番号$i$ ($0\le i \le N_\text{s}-1$)をつける。
$0$番目の砂粒が一番下で、$N_\text{s}-1$番目の砂粒が一番上である。
砂粒が落下し、衝突しても、各砂粒の上下の位置関係が変わることはない。


シミュレーション開始時にはすべての砂柱をオリフィス面$\mathrm{O}$よりも上方に配置し、
全部で$N_0 = N_\text{p} N_\text{s}$個の砂粒の運動方程式を解く。


$p$番目の砂柱の中の$i$番目の砂粒の$y$座標を$y^p_i$と書くと、その運動方程式は、
%==========
\begin{equation} \label{250919171805} 
   m \frac{\mathrm{d}^2 y^p_i}{\mathrm{d} t^2} = -mg + R^p_{1i} + R^p_{2i}  - F^p_{i,i+1} + F^p_{i,i-1}
\end{equation}
%==========
である。
ここで $R^p_{1i}$, $R^p_{2i}$, $F^p_{i,i+1}$, $F^p_{i,i-1}$は、
それぞれ砂柱$p$がオリフィス面$\mathrm{O}$から受ける抗力、底面$\mathrm{B}$から受ける抗力、
上の砂粒と接触した場合に受ける下向きの力(バネ力とダンパー力の和)、
下の砂粒と接触した場合に受ける上向きの力である。


ある一本の砂柱の中で、オリフィス面 $\mathrm{O}$と接触している粒子は常に0個または1個である。
粒子$i$が$\mathrm{O}$と接触していなければ式\eqref{250919171805}で$R^p_{1i}=0$とする。
底面$\mathrm{B}$についても同様である。


各砂柱は、オリフィス面$\mathrm{O}$と接触する可能性のある粒子番号を保持している。
シミュレーション開始時にはその番号は$i=0$(砂柱の中で一番下に位置する砂粒)である。
シミュレーション開始時に$i=0$の砂粒が$\mathrm{O}$から半径$a$よりも大きな距離だけ離れている、つまり接触していない場合でも、
時間の経過と共にこの粒子0は落下し、いずれ$\mathrm{O}$と接触する。
この粒子0は$\mathrm{O}$から上向きの力$R^p_{1i}$の力を受けて減速し、やがて静止するが、
静止する前、あるいはまだ動いている間に、上方から別の粒子$i=1$が落下し、衝突して下向きの力$-F^p_{01}$を粒子0に及ぼす。
そしてこの粒子1もまたその上の粒子2によって上から押さえつけられる。


十分に時間が経過すると、この砂柱の粒子はすべて静止する。
この状態では、一番下の粒子0は自分自身の重みに加えて、上に積まれた$(N_\text{s}-1)$個の粒子の重みによって下方に押さえつけられ、
その力はオリフィス面$\mathrm{O}$とのバネ力$R^p_{10}$とつり合っている。
つまり$R^p_{10}=N_\text{s} mg$である。
オリフィス面$\mathrm{O}$が砂柱$p$を支える抗力がこの$R^p_{10}$である。


% . . . . . . . . . . . . . . . . . . . . . . . . . . . . . . . . . . . . . . . . 
      \marginnote{図\ref{250921063433}}
% . . . . . . . . . . . . . . . . . . . . . . . . . . . . . . . . . . . . . . . . 
オリフィスを通過して落下する砂粒の数の時間的な割合、すなわちフラックス$\mu$を一定にするために以下の方法を用いた。
一本の砂柱の砂粒には0番から順番に上に向かって番号がつけてあるので、
当初は粒子0は上方で粒子1と、下方で床面と相互作用する(図\ref{250921063433}左)。
床面が粒子0を支えることで、全ての粒子が床面の上に直線上に積み重なった状態で静止する。
あらかじめ指定した一定の時間$\mu^{-1}$が経過すると、
床面と相互作用する粒子が1に切り替わる。
その結果、下から支える力がなくなった粒子0は(床面をすり抜けて)下方に落下する(図\ref{250921063433}中)。
粒子1は床面と衝突するとダンパーによる緩和を経て静止する(図\ref{250921063433}右)。
以下同様に$\mu^{-1}$の時間が経過するごとに床面と相互作用する粒子の番号が切り替わる。
ここでは$y=0$にあるオリフィス面$\mathrm{O}$について説明したが、$y=-H_0$にある下の底面$\mathrm{B}$についても同様に粒子0は$\mathrm{B}$と衝突し、バネ・ダンパーモデルによる緩和を経て静止する。
下の底面$\mathrm{B}$は常に粒子0と相互作用するので、粒子0が(したがって他の粒子も)さらに下に落ちることはない。


% . . . . . . . . . . . . . . . . . . . . . . . . . . . . . . . . . . . . . . . . 
      \marginnote{図\ref{250919165122}}
% . . . . . . . . . . . . . . . . . . . . . . . . . . . . . . . . . . . . . . . . 
オリフィス面$\mathrm{O}$が相互作用する相手の粒子番号を周期$\mu^{-1}$で切り替えるわけであるが、
$N_\mathrm{P}$本ある砂柱でその切り替えの位相をずらすことで砂粒の落下をバラバラにする。
位相をずらす単位は$\Delta \psi = 2\pi/N_\mathrm{P}$とした。
$p$番目の砂柱の位相を$p\Delta \psi $とすると、水平方向に並んだ砂柱から順番に次々と砂が落下していくので見た目は綺麗ではあるものの、砂時計というには違和感があるので、それぞれの砂柱にはランダムな順番で$N$個の位相のずれを割り当てた。
もちろんこれは見た目だけの問題でシミュレーション結果には無関係である。
図\ref{250919165122}は$N_\mathrm{P}=14$で各砂柱中の粒子数$N_\mathrm{S}=20$のときのシミュレーションのスナップショットである。



%------------------------------------------
\subsection{シミュレーション結果}\label{250923145905}
%------------------------------------------


% . . . . . . . . . . . . . . . . . . . . . . . . . . . . . . . . . . . . . . . . 
      \marginnote{図\ref{250921210605}}
% . . . . . . . . . . . . . . . . . . . . . . . . . . . . . . . . . . . . . . . . 
図\ref{250921210605}に$N_\mathrm{P}=24$, $N_\mathrm{S}=40$の結果を示す。
グラフの横軸は時間、縦軸は砂を支える抗力$F$を重さで表した値($F/g$)である。
$N_0=N_\mathrm{P}\times N_\mathrm{S}=960$, 
$H_0=2.0$~\si{(m)},
$K_0=1.0$~\si{(m)},
$M_0=1.0$~\si{(kg)},
$T_0=10.0$~\si{(s)}とした。
$t=-2~\si{(s)}$にシミュレーションを開始し、
$t=0$に砂の落下を開始する。
式\eqref{250919160045}できまる時間刻み幅は $\Delta t = 1.996\times 10^{-6}$~\si{(s)} である。


$t=-2$の初期条件では、各砂粒を上のオリフィス面$\mathrm{O}$よりも上の座標に適当な位置に配置する。
シミュレーションが開始すると、各粒子は重力によって下方に移動し、
各粒子が自重と自分よりも上方にある粒子の重みによってそれぞれのバネがわずかに縮んだ状態で力学的平衡状態に達する。
この平衡状態は初期の粒子配置の仕方には依存しない。
砂粒同士をちょうど直径$2a$だけ離して順番に積み上げた状態で配置すると、
砂粒の重みによってわずかに下方に移動し、すぐに平衡状態に達して$t=0$よりも十分以前に平衡状態に達する。


時刻$t=0$にオリフィス面$\mathrm{O}$が砂を支える抗力$R_1$は、
各砂柱の一番下($i=0$)の砂粒と上のオリフィス面$\mathrm{O}$の間で作用する力の和である。
%==========
\begin{equation} \label{250927163046} 
   R_1(t=0) = \sum_{p=0}^{N_\mathrm{P}-1} R^p_{10}
\end{equation}
%==========
これは砂全体にかかる重力$M_0 g$に等しい。
時間が経過すると、オリフィス面$\mathrm{O}$と相互作用する粒子は0番から1番、1番から2番へと切り替わる。
ある時刻$t>0$において上のオリフィス面$\mathrm{O}$と相互作用する砂柱$p$の粒子番号を$u_p$とすると、
%==========
\begin{equation} \label{250927163759} 
   R_1(t) = \sum_{p=0}^{N_\mathrm{P}-1} R^p_{1u_p}
\end{equation}
%==========
である。


底面$\mathrm{B}$は常に0番の粒子と相互作用するので、床面$\mathrm{B}$が(落下して下に堆積した)砂を支える抗力$R_2$は
%==========
\begin{equation} \label{250927164134} 
   R_2(t) = \sum_{p=0}^{N_\mathrm{P}-1} R^p_{20}
\end{equation}
%==========
である。


図\ref{250921210605}の縦軸は、二つの床面にはたらく力の和$F(t)=R_1(t)+R_2(t)$である。
$F(t)$は時間変動が大きいので、
左のグラフでは0.2~\si{(s)}毎の平均値をとっている。% 正確には 1万ステップ毎の平均値


左のグラフの$t<0$の測定値は$W=R/g=M_0=1$の線に乗っている。
$t=0$に砂の落下が始まると砂時計の重さは変動し、最初の砂粒が$y=-H_0$の底面に$\mathrm{B}$に到達するまでは総じて軽くなる。
その後、$t\sim 2$から$t=T_0=10$まではやはり変動が大きいものの、平均をとれば$W=1$よりも大きくなる。
$t=10$以降の短い時間、落下する砂粒による力積の効果により$W$はさらに大きくなる。


シミュレーションで得られた砂時計の重さの変化$\Delta W$をより細かくみるために、
右のグラフでは0.48~\si{(s)}毎の平均値をとり、縦軸を拡大してプロットした。 
%
% 正確には24万ステップ毎の平均値
% 
式\eqref{250703173052}による理論値は
$\Delta W=2M_0 K_0/gT_0^2 = 2.039\times 10^{-3}$~\si{(kg)}である。
シミュレーションで得られた重さの増加は理論とよく一致していることがわかる。


%=============================================
\section{まとめ}
%=============================================

砂が落ちている間、砂時計はわずかに重くなる。 
このよく知られている結果を可変質量系の力学の問題として定式化し、運動方程式を用いて導出して重みが増加する起源を明らかにした。


砂が落ちるにつれて上の砂層は薄くなる。
上の砂層の重心は下がるので、上の砂層全体がもつ運動量$P_1$は下向きであるが、その時間微分$\dot{P}_1$は上向きである。
一方、下の砂層は時間の経過と共に厚くなり、その運動量$P_2$と$\dot{P}_2$はどちらも上向きである。
全体として、計時中の砂時計の重心は常に下に移動するが、
加速度は上向きである。
つまり重心の軌跡を時間の関数としてプロットすると、自由落下の場合のような上に凸の放物線ではなく、下に凸な放物線になる。



砂の層の厚みの変化にともなう運動量の時間変化の効果に加えて見逃し易いが決して無視できない大きさを持つ効果は、
下の層が厚くなるにつれ、砂の落下距離が短くなることから、落下中の砂がもつ下向きの運動量$P_3$が時間変化する効果$\dot{P}_3>0$である。この効果も上向きである。


砂時計の重さの増分が増える理由は、これら3つの和で、その値は$\Delta F=2M_0K_0/T_0^2$である。
手のひらに乗るほどの砂時計を想定し、砂が落ちる前の砂の層の厚さ$K_0$を$\mathcal{O}(10^{-1})$~\si{(m)}、
砂時計の計時時間$T_0=\mathcal{O}(10^{2})~\si{(s)}$とすると、
砂の質量$M_0$にかかる重力$M_0 \, g$と比較した$\Delta F$の相対的な大きさは$2K_0/gT_0^2=\mathcal{O}(10^{-5})$と極めて小さいので、
人が感じることはできないであろう。
なお、砂時計の重さの変化は、$H_0$(砂の落下距離)には依存しない。


このようなわずかな変化をコンピュータシミュレーションで定量的に再現するために、
我々はバネ=ダンパーモデル(ソフトスフィアモデル)に基づく粒子法シミュレーションを行った。
理論的な解析が示すように、砂時計の重さの変化を引き起こすのは、
砂粒が鉛直下方に移動する自由落下運動と、上の砂層の厚みと下の砂層の厚みが時間的に変化することに起因する。
したがって砂粒の水平方向の運動は本質的ではないので、鉛直方向のみを扱う1次元的な粒子シミュレーションでこの現象を定量的に再現できるはずと考えた。
1次元の粒子モデルでは、一つの粒子は最大でも上下二つの隣の粒子とだけ接触し、しかも接触する相手は変わらないので、取り扱いが単純で計算が高速になる。
%複数の粒子が同時に一つの粒子に接触するような状況でも高精度に追跡できることがソフトスフィアモデルの利点である。



1次元粒子シミュレーションには上述の利点があるが、
砂の柱が1本だけしかないようなシミュレーションモデルでは与えられた時間の間に多数の砂を落下させることができない。
そこで本研究では複数の砂柱からそれぞれ独立に砂粒が落下するモデルを考案した。
これにより十分な数の砂粒を落下させることができ、理論的に予測される重さの増分を定量的に再現することに成功した。
これは通常の砂時計のようなくびれのある容器ではなく、円筒形の容器の真ん中に多数の穴のある板を差し込んだTuinstra等の実験においてなされたものと同じ工夫である\cite{Tuinstra2010-wk}。


本研究で開発したシミュレーションプログラムはProcessing言語で書き、ノートPCで実行した。
この言語では、単精度実数の浮動小数点数変数しかデフォルトで用意されていないが、
本研究では倍精度浮動小数点数で計算した。
%シミュレーションプログラムはGitHubで公開する。


このシミュレーションの特徴は、扱うべきマクロな時間スケールとミクロな時間スケールが大きく異なることである。
ミクロな時間スケールは、粒子の衝突で決まる。
自由落下速度$v_0$で落下してきた半径$a$の砂粒が、下の砂粒に衝突したことを検知するためには、
シミュレーションの時間刻み幅$\Delta t$が$\Delta t\ll a/v_0$でなければならない。
第\ref{250923145905}節で紹介したシミュレーションでは、$\Delta t = 1.996\times 10^{-6}$~\si{(s)} とした。
一方、マクロな時間スケールは砂時計の計時時間$T_0$である。
計時時間$T_0$を$10$~\si{(s)} 程度とした場合、
シミュレーションの時間積分ステップ数は少なくとも$T_0 / \Delta t=5\times 10^6$は必要である。
とはいえこの程度の数値積分はノートPCでも十分に実行でき、Processingがもつ可視化機能によりリアルタイムで砂の落下を観察することもできる。





\bibliographystyle{amsplain} 
\bibliography{hourglass_weight}

\appendix

%=============================================
\section{下の砂層の厚さの時間変化}\label{250919092352}
%=============================================

物体3は自由落下する砂で構成される。
その砂粒の総数は、$y=0$から下の砂層$K_2(t)$の最上部$y=-H_0+K_2(t)$まで自由落下する時間$\tau_f(t)$を使い、
%==========
\begin{equation} \label{250917175724} 
   N_3(t) = \mu \tau_f(t) 
    = \mu \sqrt{\frac{2(H_0-K_2(t))}{g}} 
    % = \mu \sqrt{\frac{2}{g}}\sqrt{H_0-K_2(t)}
\end{equation}
%==========
と書ける。
したがって物体3の質量は$M_3(t) = mN_3(t) $である。
%%==========
%\begin{equation} \label{250917175839} 
%   M_3(t) = mN_3(t) %= m \mu \sqrt{\frac{2(H_0-K_2(t))}{g}} 
%   		 = m\mu\sqrt{\frac{2}{g}}\sqrt{H_0-K_2(t)}
%\end{equation}
%%==========
%%==========
%\begin{equation} \label{250924142352} 
%   K_3(t) = \frac{m\mu}{\sigma}\sqrt{\frac{2}{g}}\sqrt{H_0-K_2(t)}
% \end{equation}
%%==========





砂の総質量
%==========
\begin{equation} \label{250917175922} 
   M_0 = M_1(t) + M_2(t) + M_3(t)
\end{equation}
%==========
は定数なので、この式の時間微分をとり、式\eqref{250919084914}を使うと、
%これは
%%==========
%\begin{equation} \label{250919085544} 
%   \sigma K_0 =  \sigma K_1(t) + \sigma K_2(t) + m\mu\sqrt{\frac{2}{g}}\sqrt{H_0-K_2(t)}
%\end{equation}
%%==========
%%==========
%\begin{equation} \label{250920140526}
%   K_0 =  K_1(t) + K_2(t) + \frac{m\mu}{\sigma}\sqrt{\frac{2}{g}}\sqrt{H_0-K_2(t)}
%\end{equation}
%%==========
%この式の時間微分をとる。
%%==========
%\begin{equation} \label{250920140735} 
%   0 = \dot{K}_1(t) +   \frac{\mathrm{d}}{dt}\left\{K_2(t)  + \frac{m\mu}{\sigma}\sqrt{\frac{2}{g}} \sqrt{H_0-K_2(t)} \right\}
%\end{equation}
%%==========
%式\eqref{250919084914}より
%%==========
%\begin{equation} \label{250919085137} 
%    0 =  -\frac{m\mu}{\sigma} +   \dot{K}_2(t)  + \frac{m\mu}{\sigma}\sqrt{\frac{2}{g}} \frac{\mathrm{d} }{\mathrm{d} t}\sqrt{H_0-K_2(t)}
%\end{equation}
%%==========
%両辺を$H_0$で割ると
%==========
\begin{equation} \label{250920140957} 
    0 =  -\frac{m\mu}{\sigma H_0} +   \frac{\dot{K}_2(t)}{H_0}  + \frac{m\mu}{\sigma}\sqrt{\frac{2}{gH_0}} \frac{\mathrm{d} }{\mathrm{d} t}\sqrt{1-\frac{K_2(t)}{H_0}}
\end{equation}
%==========
となる。ここで、$K_2(t)$を$H_0$で規格化した
%==========
\begin{equation} \label{250920141535} 
   k_2(t) = \frac{K_2(t)}{H_0}
\end{equation}
%==========
と、時間の逆数の次元をもつ定数
%==========
\begin{equation} \label{250920141440} 
   \beta = \frac{m\mu}{\sigma H_0}
\end{equation}
%==========
および無次元定数
%==========
\begin{equation} \label{250920141156} 
   \epsilon = \frac{m\mu}{\sigma}\sqrt{\frac{2}{gH_0}} = 2\frac{K_0/T_0}{v_0}
\end{equation}
%==========
を定義する。
式\eqref{250920141156}の最右辺は、砂層が変化する速度と砂粒の(最大の)自由落下速度の比(の2倍)である。
したがって$\epsilon$は微少量である。
$k_2(t)$に対する微分方程式
%==========
\begin{equation} \label{250920141635} 
    \beta = \frac{\mathrm{d} }{\mathrm{d} t} \left( k_2(t)  + \epsilon \sqrt{1-k_2(t)}\right)
\end{equation}
%==========
を$t=0$で$k_2=0$の初期条件の下で解くと、
%==========
\begin{equation} \label{250920151147} 
  k_2(t)  = \beta t +  \epsilon\left(1-\sqrt{1-\beta t}\right)
			+ \mathcal{O}(\epsilon^2)
\end{equation}
を得る。
この式を微分すると、
%==========
\begin{equation} \label{250920152117} 
   \dot{k}_2(t) = \beta + \frac{\beta \epsilon}{2\sqrt{1-\beta t}}+ \mathcal{O}(\epsilon^2)
\end{equation}
%==========
である。
なお、$1-\beta t \ge 1-\beta T_0 = 1-K_0/H_0>0$である。


%式\eqref{250920151147}を$K_2$で書き、
%初期条件を$t=0$から$t=\tau_0$にずらすと
%$ \mathcal{O}(\epsilon)$までの近似解が
%%==========
%\begin{equation} \label{250920170419}
%  K_2  = \frac{m\mu}{\sigma} (t-\tau_0) +  \epsilon H_0 \left(1-\sqrt{1-\beta  (t-\tau_0)} \right)
%			+ \mathcal{O}(\epsilon^2)
%\end{equation}
%%==========
%である。また
式\eqref{250920152117}を$\dot{K}_2$で書き直し、初期条件を$t=0$から$t=\tau_0$にずらすと
%==========
\begin{equation} \label{250920170423}
   \dot{K}_2  = \frac{m\mu}{\sigma}\left( 1+ \frac{\epsilon}{2\sqrt{1-\beta  (t-\tau_0)}}\right)+ \mathcal{O}(\epsilon^2)
\end{equation}
%==========
を得る。



\newpage
%=============================================
\section*{図表一覧}
%=============================================
\begin{description}
  % fig01
  \item[図\ref{250919103231}]
    砂時計内部の砂を支える抗力と本研究における砂のモデル。
    砂は質量が変化する物体と考える。
    落下前の上の砂層を物体1、落下後の下の砂層を物体2、落下中の砂粒群を物体3とする。
    砂時計の重さが変わるのは、砂時計の上の「床面」が物体1を支える抗力$R_1$と、
    下の「床面」の抗力$R_2$の和が時間変化するためである。
    
  % fig02
  \item[図\ref{251120173401}]
	砂粒のモデル。(a) 全ての砂粒は半径$a$、質量$m$の球とし、
      内部にバネ・ダンパーのペアを二つもつ。バネの自然長は$a$とする。
      (b) 隣り合う二つの砂粒の中心間距離が$2a$よりも短くなった場合、
      砂粒どうしが接触していると判定する。接触したときにのみ半径の重なりに比例する反発力と、
      砂粒同士の相対速度に比例する抵抗力が発生する。
      (c) $n$個の砂粒が積み重なって静止状態にあるとき、一番下の砂粒の下側のバネは$\Delta r$だけ縮む。上向きのバネ力$F=k\Delta r$は、(自分も含めて)砂粒全体にかかる重力$nmg$とつりあう。作用反作用により$F$は床からの抗力$R$に等しい。

  % fig03
  \item[図\ref{250921122102}]
    砂粒の落下テスト。
    半径$a=1\times 10^{-2}$~\si{(m)}の砂粒を、高さ$y=1~\si{(m)}$から自由落下させる。
    $y=0$に設定した床面に非弾性衝突して静止する。
    横軸は時間、縦軸は粒子の中心の$y$座標。
    右のグラフは左のグラフを衝突点付近で拡大したもの。
    $y=a$で床面に接触し、粒子のバネとダンパーによって衝突のエネルギーが吸収される。

  % fig04
  \item[図\ref{250924103856}]
    砂柱モデル。砂の「柱」が複数あり、それぞれで砂粒が鉛直方向のみに運動するモデル。
    砂の「柱」が一本だけの場合(左の図)と比較して、単位時間あたりに落下する砂粒の数を増加させることができる(右の図)。

  % fig05
  \item[図\ref{250921063433}]
    オリフィスのモデル。
    時間の経過と共に一定の割合$\mu~\si{(s^{-1})}$で砂粒が落下する粉体特有の現象をソフトスフィアモデルに組み込むために、
    本シミュレーションでは床面との接触(抗力)を担う粒子を一定の割合で切り替える。
    左の図では粒子0の下側のバネ(マゼンタ色)が床面と接触し、抗力を受けている。
    $1/\mu$~\si{(s)}の時間が経過すると、床面と相互作用する粒子は1に切り替わる(真ん中の図)。
    粒子0は床に支えられていないので、下方に落下する。
    粒子1(およびそれよりも上の粒子2等)は重力によって落下するが、
    すぐに床面に衝突して次の切り替え時刻まで、床面に支えられる(右の図)。

  % fig06
  \item[図\ref{250919165122}] 
    砂柱の数$N_\mathrm{P}=14$で、各砂柱中の粒子数$N_\mathrm{S}=20$でのシミュレーションのスナップショット。
    $\mathrm{O}$はオリフィス面、$\mathrm{B}$は底面を示す。
    砂が落ちる前(左の図)、
    落ちている最中(真ん中の図)、
    全ての砂が落ちた後(右の図)。

  % fig07
  \item[図\ref{250921210605}] 
    質量$M_0=1$~\si{(kg)}の砂を支える抗力$F$の時間変化を重さ$W=F/g$で示したグラフ。
    左のグラフは時間間隔$0.2$~\si{(s)}で平均をとった。
    砂の落下は$t=0$から$t=10$~\si{(s)}までである。
    砂の落下前後、砂が静止している状態では$W=1$の線上にデータが乗っている。
    右のグラフは時間間隔$0.48$~\si{(s)}で平均をとり、縦軸を拡大してプロットしたもの。
    砂の重さの増分$\Delta W$の理論値$2.039\times10^{-3}$~\si{(kg)}は``theory''とラベルをつけた水平線で示した。
\end{description}



\newpage
%=========================================================================
\section*{著者紹介}
%=========================================================================

\noindent
陰山 聡(会員)

神戸大学システム情報学研究科教授。
広島大学大学院理学研究科修了。
核融合科学研究所理論・シミュレーション研究センター、海洋研究開発機構地球シミュレータセンター、神戸大学工学研究科を経て2010年より現職。
主に地磁気に関連した計算科学と可視化を含めたシミュレーション手法の研究を行っている。


\vspace{1em}
\noindent
中島 涼輔(非会員)

神戸大学システム情報学研究科特命助教。
九州大学大学院理学府地球惑星科学専攻博士後期課程修了。
九州大学大学院理学研究院特任助教を経て2025年より現職。
主にシミュレーションを用いて、地球内部物理学、地球電磁気学、地球流体力学に関する研究を行っている。


\vspace{1em}
\noindent
中戸 昂明(非会員)

神戸大学システム情報学部学生。



\newpage
%-------------------------------------------------------
\begin{figure}[ht]   \centering   
  \includegraphics[%
     height=1.0\textheight,%
       width=0.8\hsize,keepaspectratio]%
         {./figs/schematic_objects_and_forces.png}   
      \caption{砂時計内部の砂を支える抗力と本研究における砂のモデル。
      砂は質量が変化する物体と考える。
      落下前の上の砂層を物体1、落下後の下の砂層を物体2、落下中の砂粒群を物体3とする。
      砂時計の重さが変わるのは、砂時計の上の「床面」が物体1を支える抗力$R_1$と、
      下の「床面」の抗力$R_2$の和が時間変化するためである。}
      \label{250919103231}
\end{figure} 
%-------------------------------------------------------


\newpage
%-------------------------------------------------------
\begin{figure}[ht]   \centering   
  \includegraphics[%
     height=1.0\textheight,%
       width=0.8\hsize,keepaspectratio]%
         {./figs/softSphereModel.png}   
      \caption{砂粒のモデル。(a) 全ての砂粒は半径$a$、質量$m$の球とし、
      内部にバネ・ダンパーのペアを二つもつ。バネの自然長は$a$とする。
      (b) 隣り合う二つの砂粒の中心間距離が$2a$よりも短くなった場合、
      砂粒どうしが接触していると判定する。接触したときにのみ半径の重なりに比例する反発力と、
      砂粒同士の相対速度に比例する抵抗力が発生する。
      (c) $n$個の砂粒が積み重なって静止状態にあるとき、一番下の砂粒の下側のバネは$\Delta r$だけ縮む。上向きのバネ力$F=k\Delta r$は、(自分も含めて)砂粒全体にかかる重力$nmg$とつりあう。作用反作用により$F$は床からの抗力$R$に等しい。}
      \label{251120173401}
\end{figure} 
%-------------------------------------------------------

\newpage
%-------------------------------------------------------
\begin{figure}[ht]   \centering   
  \includegraphics[%
     height=1.0\textheight,%
       width=1.0\hsize,keepaspectratio]%
         {./figs/fallingSingleGrain.png}   
      \caption{砂粒の落下テスト。
      半径$a=1\times 10^{-2}$~\si{(m)}の砂粒を、高さ$y=1~\si{(m)}$から自由落下させる。
      $y=0$に設定した床面に非弾性衝突して静止する。
      横軸は時間、縦軸は粒子の中心の$y$座標。
      右のグラフは左のグラフを衝突点付近で拡大したもの。
      $y=a$で床面に接触し、粒子のバネとダンパーによって衝突のエネルギーが吸収される。}
      \label{250921122102}
\end{figure} 
%-------------------------------------------------------

\newpage
%-------------------------------------------------------
\begin{figure}[ht]   \centering   
  \includegraphics[%
     height=1.0\textheight,%
       width=0.5\hsize,keepaspectratio]%
         {./figs/sandPileModel.png}   
      \caption{砂柱モデル。砂の「柱」が複数あり、それぞれで砂粒が鉛直方向のみに運動するモデル。
      砂の「柱」が一本だけの場合(左の図)と比較して、単位時間あたりに落下する砂粒の数を増加させることができる(右の図)。}
      \label{250924103856}
\end{figure} 
%-------------------------------------------------------

\newpage
%-------------------------------------------------------
\begin{figure}[ht]   \centering   
  \includegraphics[%
     height=1.0\textheight,%
       width=0.5\hsize,keepaspectratio]%
         {./figs/schematic_orifice.png}   
      \caption{オリフィスのモデル。
      時間の経過と共に一定の割合$\mu~\si{(s^{-1})}$で砂粒が落下する粉体特有の現象をソフトスフィアモデルに組み込むために、
      本シミュレーションでは床面との接触(抗力)を担う粒子を一定の割合で切り替える。
      左の図では粒子0の下側のバネ(マゼンタ色)が床面と接触し、抗力を受けている。
      $1/\mu$~\si{(s)}の時間が経過すると、床面と相互作用する粒子は1に切り替わる(真ん中の図)。
      粒子0は床に支えられていないので、下方に落下する。
      粒子1(およびそれよりも上の粒子2等)は重力によって落下するが、
      すぐに床面に衝突して次の切り替え時刻まで、床面に支えられる(右の図)。
      }
      \label{250921063433}
\end{figure} 
%-------------------------------------------------------

\newpage
%-------------------------------------------------------
\begin{figure}[ht]   \centering   
  \includegraphics[%
     height=1.0\textheight,%
       width=1.0\hsize,keepaspectratio]%
         {./figs/fallingSands.png}   
      \caption{砂柱の数$N_\mathrm{P}=14$で、各砂柱中の粒子数$N_\mathrm{S}=20$でのシミュレーションのスナップショット。
      $\mathrm{O}$はオリフィス面、$\mathrm{B}$は底面を示す。
      砂が落ちる前(左の図)、
      落ちている最中(真ん中の図)、
      全ての砂が落ちた後(右の図)。
      }
      \label{250919165122}
\end{figure} 
%-------------------------------------------------------

\newpage
%-------------------------------------------------------
\begin{figure}[ht]   \centering   
  \includegraphics[%
     height=1.0\textheight,%
       width=1.0\hsize,keepaspectratio]%
         {./figs/sandWeight.png}   
      \caption{質量$M_0=1$~\si{(kg)}の砂を支える抗力$F$の時間変化を重さ$W=F/g$で示したグラフ。
      左のグラフは時間間隔$0.2$~\si{(s)}で平均をとった。
      砂の落下は$t=0$から$t=10$~\si{(s)}までである。
      砂の落下前後、砂が静止している状態では$W=1$の線上にデータが乗っている。
      右のグラフは時間間隔$0.48$~\si{(s)}で平均をとり、縦軸を拡大してプロットしたもの。
      砂の重さの増分$\Delta W$の理論値$2.039\times10^{-3}$~\si{(kg)}は``theory''とラベルをつけた水平線で示した。
      }
      \label{250921210605}
\end{figure} 
%-------------------------------------------------------

\end{document}
