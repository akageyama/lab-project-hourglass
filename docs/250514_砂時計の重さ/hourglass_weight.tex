\documentclass[]{article}
\usepackage{siunitx}
\usepackage[dvipdfm]{hyperref}
\usepackage[dvipdfmx]{graphicx}
\usepackage{bm}
\usepackage{fancyhdr}
\usepackage{indentfirst}
\usepackage{listings}
\usepackage{amsmath,amssymb}
\usepackage{here}
\usepackage{ascmac}
\usepackage[dvipsnames]{xcolor}
\usepackage{url}
\usepackage{colortbl}
\usepackage{comment}

\title{砂時計の重さ\footnote{%
これは学部の1年生向けの講義で力積について教えた際に例題の一つとして挙げようとした話題である。
力積の計算から「重さは変わらない」という結論がでることがちょっと面白いであろうと思って準備をしていたが、
重心の移動を考えるとそれはおかしいということに講義の直前になって気がついた。
面白い問題であることは間違いない。
}}
\author{陰山 聡\\[0.5em] 神戸大学システム情報学研究科}
\date{2025.05.18}

\begin{comment}
\end{comment}

\begin{document}

\maketitle


\begin{abstract}
砂時計の砂が落ちている間、砂時計の重さは変わるであろうか?
落ちている最中の砂粒の重みは砂時計には作用しないので、その分だけ軽くなる一方、
下に落ちた砂は床面に力積を与えるのでその分だけ重くなる。
計算するとこの両者はちょうどキャンセルするので、重さは変わらないという結論になりそうであるが、
もう少し正確に計算すると、砂時計全体はわずかに重くなる。
\end{abstract}


%=============================================
\section{問題設定}
%=============================================
$T_0$秒間を測る砂時計を考える。3分計であれば$T_0=180$~(\si{s})である。
重力加速度を$g~(\si{m.s^{-2}})$、砂粒の総数を$N$、砂全体の質量を$M_0~(\si{kg})$とする。


砂粒一つの質量$m$は、
%==========
\begin{equation} \label{250517100232} 
   m = \frac{M_0}{N} 
\end{equation}
%==========
1秒間に落ちる砂粒の数を$\mu~(\si{s^{-1}})$とすると、
%==========
\begin{equation} \label{250515082817} 
   \mu = \frac{N}{T_0}
\end{equation}
%==========
である。
なお、下に落ちた砂粒は跳ね返らず、短い時間で静止すると仮定する。


%=============================================
\section{力積の評価による概算} \label{250517204115}
%=============================================

一つの砂粒が砂時計のくびれ部分(オリフィス)から落下し、下部に積み重なった砂の層の表面まで到達するのにかかる時間、つまり自由落下時間を$\tau_f$~(\si{s})とすると、
落下した瞬間の砂粒の速さ(自由落下速度)は
%==========
\begin{equation} \label{250512190520} 
   V_F = g\tau_f
\end{equation}
%==========
なので、落下によって静止した砂粒の運動量の変化、つまり力積は
%==========
\begin{equation} \label{250512190655} 
   p = mg\tau_f
\end{equation}
%==========
である。


$\Delta t$秒間に砂粒は$\mu \Delta t$個だけ落下する。
この$\Delta t$秒間に砂粒が下面を押す力の平均を$F_\mathrm{i}$~(\si{N})とすると
%==========
\begin{equation} \label{250512191540} 
  \Delta t \times F_\mathrm{i} = \mu \Delta t \times p = \mu \Delta t \times mg\tau_f
\end{equation}
%==========
つまり
%==========
\begin{equation} \label{250515093535} 
  F_\mathrm{i}   = \mu m g \tau_f
\end{equation}
%==========
である。


一方、ある瞬間に落下している途中、つまり砂時計のオリフィスから下面までの中空にいる砂粒は$\mu \tau_f$個ある。
一つの砂粒にかかる重力は$mg$なので、
砂時計全体の重さはこの落下中の砂粒の重さ
%==========
\begin{equation} \label{250513083619} 
   F_\mathrm{s} = \mu \tau_f\times mg 
\end{equation}
%==========
だけ軽くなる。
これは力積による平均力、つまり式~\eqref{250515093535}と等しい。
従って砂が落ちている最中の砂時計全体の重さは変わらない、という結論になりそうだが、厳密にはこれは正しくない。


砂時計全体の重心の運動について考えよう。
砂が落ち始める前と落ちきった後とでは明らかに砂時計全体の重心位置は下に移動している。
したがって重心には、はじめに下向きの力が作用し、その後、上向きの力でその下方向の速度を止めたはずである。
ただし、その重心移動の距離はわずかで、しかもかなり時間をかけたゆっくりとしたもの(砂時計が3分計であれば3分間)なので、
その加速度は必然的に小さく、したがって砂時計全体の重さに与える影響は小さいことは予想できる。


%=============================================
\section{運動方程式に基づいた計算}
%=============================================

砂時計の重心が移動する効果を考慮するためにはこの問題を静力学の問題としてではなく、運動方程式に基づいて考えるべきである。


鉛直上向きに$y$軸をとり、
オリフィスの位置を$y=0$とする。
オリフィスから砂時計のガラスの底面までの距離、
つまりガラス内部の空間の高さの半分を$H_0$~(\si{m})とする。
砂が移動するガラス内部の天井面と床面の位置は
それぞれ$y=H_0$と$y=-H_0$である。


オリフィス付近を除いた砂時計の断面積を$S_0$~(\si{m^2})、
計時を開始する前の砂の層の厚さを$K_0$~(m)、
砂粒の質量密度を$\rho$~(\si{kg/m^3})とすると
砂の全質量が$M_0$~(\si{kg})なので
%==========
\begin{equation} \label{250703152145} 
   M_0 = N m = S_0 K_0 \rho
\end{equation}
%==========
である。


計時開始時刻を$t=0$とすると、
時刻$t$では$\mu t$個の砂粒がオリフィスから落下しているので、
この時刻にオリフィスの上の部分にある砂、つまりまだ落下していない砂の数は
%==========
\begin{equation} \label{250703152453} 
   N_1(t) =  N - \mu t
\end{equation}
%==========
である。
したがってオリフィス上部の砂の質量は
%==========
\begin{equation} \label{250703152655} 
   M_1(t) = m (N-\mu t)
\end{equation}
%==========
である。
この時刻におけるオリフィス上部の砂層の厚さは
%==========
\begin{equation} \label{250703152748} 
   K_1(t) = \frac{M_1(t)}{\rho S_0} = \frac{m}{\rho S_0}(N-\mu t)
\end{equation}
%==========
である。
この層の重心のy座標を$Y_1(t)$とすると
%==========
\begin{equation} \label{250703153013} 
   Y_1(t) = \frac{K_1(t)}{2} = \frac{m}{2\rho S_0} ( N - \mu t)
\end{equation}
%==========


時刻$t=0$に落下した最初の砂粒が下の床に到達するまでの時間、つまり自由落下時間を$\tau_{f0}$とすると
%%==========
%\begin{equation} \label{250703153227} 
%   \tau_{f0} = \sqrt{\frac{2H_0}{g}}
%\end{equation}
%%==========
これ以降の時刻$t>\tau_{f0}$におけいて、下の床面まで落下した砂の層には
%==========
\begin{equation} \label{250703153525} 
   N_2(t) = \mu ( t - \tau_{f0})
\end{equation}
%==========
個の砂がある。
したがって下の砂層の質量は
%==========
\begin{equation} \label{250703153950} 
   M_2(t) = m\mu(t-\tau_{f0})
\end{equation}
%==========
層の厚さは
%==========
\begin{equation} \label{250703154021} 
   K_2(t) = \frac{M_2(t)}{\rho S_0} = \frac{m\mu}{\rho S_0} (t-\tau_{f0})
\end{equation}
%==========
したがって下の砂層の重心の$y$座標は
%==========
\begin{equation} \label{250703154424} 
   Y_2(t) = -H_0 + \frac{K_2(t)}{2} = -H_0 + \frac{m\mu}{2\rho S_0} (t-\tau_{f0})
\end{equation}
%==========
である。


上の砂層を質量$M_1(t)$をもつ物体1、下の砂層を質量$M_2(t)$をもつ物体2とみなし、その運動方程式を考える。
通常の物体と異なり、どちらも質量は時間的一定ではなく、
物体1の質量は式\eqref{250703152655}から常に減少しており、
物体2の質量は式\eqref{250703153950}にしたがって常に増加する。


運動方程式を立てるまえに、オリフィスから落下した砂粒が時刻$t$において物体2の上面に衝突することで作用する力を計算する。
砂粒がオリフィスから物体2に到達するまでの時間(自由落下時間)を$\tau_f(t)$とすると
%%==========
%\begin{equation} \label{250703163145} 
%   \tau_f(t) = \sqrt{\frac{2(H_0-K_2(t))}{g}}
%\end{equation}
%%==========
落下時の速度$v_f(t)$は
%==========
\begin{equation} \label{250703161618} 
   v_f(t) = g\tau_f(t)
\end{equation}
%==========
である。
物体2は1秒間に$\mu$回、この力積(下向きの力)を受ける。
その時間平均を$F_p>0$とすると
%==========
\begin{equation} \label{250703161334} 
  F_p(f)  = m v_f(t) \mu = m g \mu \tau_f(t)
\end{equation}
%==========
である。
$\mu\tau_f(t)$はこの時刻に落下途中の砂の数なので、
%==========
\begin{equation} \label{250703172127} 
   F_p(f) = \left\{M_0-M_1(t) -M_2(t)\right\} g
\end{equation}
%==========
である。




物体1と物体2の重心の速度を$V_1(t)$と$V_2(t)$とすると
%==========
\begin{align}
   V_1(t) &= \frac{\mathrm{d} Y_1(t)}{\mathrm{d} t} = - \frac{m \mu }{2\rho S_0}  \label{250703154817a} \\
   V_2(t) &= \frac{\mathrm{d} Y_2(t)}{\mathrm{d} t} = + \frac{m \mu }{2\rho S_0}   \label{250703154817b} 
\end{align}
%==========
である。




物体1と物体2の運動方程式は
%==========
\begin{align}
   \frac{\mathrm{d} }{\mathrm{d} t} \left[
   									M_1(t) V_1(t)
   								\right] &=   - M_1(t) g + F_1\label{250703155015a} \\
   \frac{\mathrm{d} }{\mathrm{d} t} \left[
   									M_2(t) V_2(t)
   								\right]  &=  - M_2(t) g -F_p(t)+ F_2 \label{250703155015b} 
\end{align}
%==========
ここで$F_1 (>0)$はオリフィスのある面が物体1を支える抗力、
$F_2 (>0)$は砂時計の下の床面が物体2を支える抗力である。
砂時計全体を支える抗力$F$はこの二つの和である。


式\eqref{250703154817a}、\eqref{250703154817b}、\eqref{250703152655}、\eqref{250703153950}より、左辺は簡単になり
%%==========
%\begin{align}
%  - \frac{m\mu}{2\rho S_0} \frac{\mathrm{d} }{\mathrm{d} t}\left[ m ( N-\mu t)\right]
%   &=  - M_1(t) g + F_1 \label{250703170847a} \\
%  + \frac{m\mu}{2\rho S_0} \frac{\mathrm{d} }{\mathrm{d} t}\left[ m\mu(t-\tau_{f0})\right] 
%  &=   - M_2(t) g -F_p(t)+ F_2 \label{250703170847b} 
%\end{align}
%%==========
%==========
\begin{align}
   \frac{m^2\mu^2}{2\rho S_0}  &=  - M_1(t) g + F_1 \label{250703170843a} \\
   \frac{m^2\mu^2}{2\rho S_0}  &=   - M_2(t) g -F_p(t)+ F_2 \label{250703170843b} 
\end{align}
%==========
である。
したがって砂時計全体を支える抗力は
%==========
\begin{align}
   F &= F_1 + F_2  \label{250703170301a} \\
   &=M_1(t) g + M_2(t) g + F_p(t) +  \frac{m^2\mu^2}{\rho S_0}   \label{250703170301b} \\
   &=  M_0 g +  \frac{m^2\mu^2}{\rho S_0}  
   			\qquad[\ \because \text{式}~\eqref{250703172127} \text{より}\ ]  \label{250703170301c} 
\end{align}
%==========
つまり計時中の砂時計は右辺第2項
%==========
\begin{equation} \label{250703173105} 
   \Delta W  = \frac{m^2\mu^2}{\rho S_0}
\end{equation}
%==========
だけ重くなる。
式\eqref{250515082817}と式\eqref{250703152145}より
%==========
\begin{equation} \label{250703173052} 
    \Delta W  =  M_0 \frac{K_0}{T_0^2}
\end{equation}
%==========
とも書ける。


%=============================================
\section{まとめ}
%=============================================
砂が落ちていないときの砂の層の厚さを$K_0$、砂の全質量を$M_0$、砂時計が測る時間を$T_0$とすると、砂が落下しているときの砂時計は、
%==========
\begin{equation} \label{250518122927} 
   \Delta W = M_0 \frac{K_0}{T_0^2}
\end{equation}
%==========
だけ重くなる。
計時中の砂時計の重心は明らかに下に移動し、重心の速度は下向きであるが、
その加速度が上向きであることがこの重さの起源である。
砂時計の高さ、つまり砂の落下する距離や砂時計のガラス部分の断面積には依存しないことは興味深い。


たとえば、$K=5$~(\si{cm}), $M_0=100$~(\si{g}), $T_0=10^2$~(\si{s}) とすると
%==========
\begin{equation} \label{250517205142} 
   \Delta W = -5\times 10^{-7}~\si{kg}
\end{equation}
%==========
であり、この重さの差はかなり小さい。



\end{document}


